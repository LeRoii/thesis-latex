
% !TeX root = ../main.tex
\begin{chineseabstract}

随着低空经济的快速发展,无人机作为其核心载体,在军事侦察、物流配送、公共安全等领域的应用日益广泛,相应地,对于无人机在自主环境感知与智能化方面的要求日益提升。机载光电系统是实现无人机智能化的关键,然而,在实际复杂动态的低空环境中,现有系统面临多重挑战。在算法功能层面,无人机独特的俯视视角导致目标尺度变化剧烈、在图像中像素占比低且常与复杂背景混杂,通用检测模型面对此类场景时性能急剧下降,航拍视角下目标频繁被建筑物、植被遮挡,现有跟踪器易丢失目标,难以保持稳定的长时跟踪。在系统层面,机载平台有严格的尺寸、重量、功耗约束,其信息处理核心必须采用高度集成化的嵌入式边缘计算模组,此类模组的算力与内存带宽与服务器相比存在数量级差距,仅能支持经过深度优化的轻量卷积神经网络运行,对于计算复杂度更高的基于Transformer的先进模型,现有边缘计算模组难以保证其高帧率实时处理,对于参数规模庞大的视觉-语言大模型(VLM),除了最新推出的高性能计算模组外,大部分的存量设备不具备部署此类模型的硬件基础。本文围绕“多光谱融合智能光电处理算法与系统设计”这一主题,从工程落地角度出发,针对现有光电系统在实际使用中遇到的痛点难点,提供一套从核心算法到工程原型的系统性解决方案。在算法层面,分别设计面向可见光与红外图像的轻量化高精度小目标检测模型,以及具备抗遮挡能力的长时单目标跟踪框架,在系统层面,设计并实现了一套集多光谱数据采集、嵌入式智能处理与实时通信于一体的工程原型系统,具体的研究工作如下。

\begin{enumerate}[wide,]
    \item 提出了面向可见光航拍图像的高性能小目标检测网络BAP-DETR。针对无人机可见光航拍图像中目标尺度多变、背景复杂、小目标检测困难的问题,本文设计了双重注意力处理模块(Bipartite Attentive Processing Block)与基于频域感知融合的双融合编码器(Dual-Fusion Encoder),通过通道分离策略实现卷积与自注意力全局建模能力的优化,双融合编码器在融合高级语义信息的同时保留关键的低层特征,增强了对小目标特征的保留能力与多尺度特征的整合效果,并结合改进的损失函数在不增加模型推理计算复杂度的前提下提升了定位精度。在多个公开数据集上的实验表明,BAP-DETR的平均检测精度与基线模型相比提升6.9\%,计算负载降低17.5\%,与最先进的可见光航拍图像检测器相比精度提升3.8\%,实现了精度与效率的平衡,为可见光航拍图像中的小目标检测提供了有效的解决方案。

    \item 提出了面向无人机红外图像的轻量化小目标检测网络MFF-DCNet。针对无人机红外图像小目标存在的低分辨率、低对比度、目标像素占比低及边缘设备算力受限等核心挑战,本文设计了深度可分离跨阶段Transformer(Depth-wise Cross-stage Transformer,DCFormer)模块与多特征聚焦(Multi-Feature Focus,MFF)颈部结构,在确保高精度的同时,实现了在嵌入式平台上的实时推理。DCFormer将深度可分离卷积与Transformer编码器结合,在降低计算复杂度的同时,增强了主干网络特征提取能力。多特征聚焦颈部结构通过构建全新的特征融合路径,有效聚合了不同层级的语义信息和空间细节。在HIT-UAV和DroneVehicle红外航拍数据集上的实验表明,MFF-DCNet的检测精度与专用的无人机航拍图像目标检测器相比提升了5.8\%,FPS提升了10\%,与最新的DETR及其衍生方法相比,AP至少提升8.2\%,FPS是RT-DETR的3倍。同时,MFF-DCNet在Nvidia Jetson Orin NX边缘模组上实现了39.6 FPS的实时性能,展现出其在资源受限环境中的实际部署能力。

    \item 提出了一种面向边缘计算设备的抗遮挡长时目标跟踪框架SKF-Tracker。针对无人机在城市场景下面临的目标被复杂地物(如建筑、高架桥)遮挡所导致的跟踪失败的问题,本文设计了一套可复用于任何基础跟踪器的抗遮挡单目标跟踪框架,以结构相似性指数为基础构建跟踪置信评估与遮挡判定模块,并结合基于卡尔曼滤波的轨迹预测以及动态模板更新策略,实现复杂环境下的稳定目标跟踪。为覆盖多样化的城市场景,本文在多个典型区域(如密集街区、交通枢纽、公园绿地)进行了大量实地无人机飞行测试,采集原始视频素材。在此基础上,从中筛选并构建了多模态无人机视角目标跟踪数据集MMUOT-1050,该数据集共包含38段高质量视频序列(20段可见光,18段红外),并对其中总计1050个目标实例(可见光353个,红外697个)进行了逐帧的边界框标注,每个目标都经历完整的遮挡后重新出现的过程。SKF-Tracker在该数据集上实现了89.37\%的可见光视频成功率和91.93\%的红外视频成功率,分别较基线方法提升14\%和11.73\%。在实时性方面,SKF-Tracker在保持高精度的同时,仍能保持31.25 FPS,其速度远超基于深度神经网络的跟踪算法,不占用边缘侧宝贵的NPU资源。为无人机在复杂环境中实现稳定长时监视任务提供了实用的解决方案。
    
    \item 设计并实现了一套完整的智能光电原型系统。集多光谱数据采集、嵌入式智能处理与实时通信于一体,系统以高集成度的嵌入式边缘计算模组作为信息处理核心,支持多路可见光与红外视频流的采集与处理。在软件层面,本文设计了一套高度模块化、可扩展的嵌入式软件框架,框架采用多种设计模式,实现了数据采集、处理与通信服务的高效协同,通过引入无锁环形缓冲区、内存池与条件变量等机制,提升了系统的实时性能。系统通过多种通信协议(TCP/IP,UDP,HTTP,串口)与上位机建立双向数据通道,实现了“人在回路”的混合协同架构。该方案在大疆M350 RTK无人机上完成了集成与实地飞行验证,证明了其从数据采集、实时处理到信息交互的全链路工作能力,这套方案不仅为本文的算法研究提供了可靠的工程验证载体,也为后续算法部署与系统集成提供了具体的技术参考。    
\end{enumerate}

\chinesekeywordstype{无人机;光电系统;边缘计算;目标检测跟踪}{应用研究}

\end{chineseabstract}

