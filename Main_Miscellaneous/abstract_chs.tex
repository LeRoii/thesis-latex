
% !TeX root = ../main.tex
\begin{chineseabstract}

自动驾驶在节约驾驶成本、提高交通效率、减少环境污染等方面拥有巨大优势,成为了学术界和工业界的热门研究课题。为了实现安全可靠、稳定高效的驾驶行为,自动驾驶车辆需要精准地预测出周围环境中交通参与者的未来行为轨迹,并规划出自身无碰撞且运动学可行的短时运动轨迹。传统轨迹预测方法无法保证长期预测的精度,严重依赖启发式设计的传统运动规划方法也无法保证其泛化性能。近年来,基于数据驱动的深度学习方法得到了快速发展,为完成预测规划任务带来了新思路。从数据输入输出的角度考虑,预测和规划都是对交通参与者的历史特征进行建模后输出未来轨迹。因此,这两项具有共性的任务均可以采用具有强大特征拟合能力的深度学习方法来完成。然而,此类方法仍然存在交通参与者异质性处理能力差,缺少概率性预测结果以及无法保证轨迹平滑性等问题,使得自动驾驶的安全性受到威胁,阻碍了自动驾驶技术进一步的发展。

本文聚焦于利用Transformer网络解决上述核心难点问题:1)如何构造更精准、更快速的实用化轨迹预测网络模型;2)如何在保证完成自动驾驶任务的前提下促使运动规划方法尽可能地减少交通违规行为。主要研究工作如下。

\begin{enumerate}[wide,]
    \item 提出了一种基于时空Transformer网络的单模态轨迹预测网络模型,弥补了之前方法只能有效预测同质交通参与者的缺陷,提高了密集交通环境下时空交互建模能力。针对之前方法对时间序列数据进行串行处理造成的记忆能力弱以及空间邻域范围设置不合理等问题,该方法采用Transformer网络并构建了全感知域的时空图模型。整个网络包括时空Transformer编码器、时间Transformer编码器和时间Transformer解码器三个部分。时空Transformer编码器能够对时空图特征按照不同维度交替提取,从而充分融合时空信息。经过时间Transformer编码器对于时间信息的进一步处理后,时间Transformer解码器生成了关于异质交通参与者的单模态轨迹。在自动驾驶轨迹预测公开数据集上的实验结果表明,该方法比当时最好的方法在主要性能指标上提高了至少7.2\%。

    \item 提出了一种基于概率性候选轨迹网络的多模态轨迹预测网络模型,在加快模型推理速度的同时,提高了多模态轨迹预测的精度。针对当前多模态轨迹预测方法无法提供概率性预测结果的问题,该方法设计了一种既能生成目标点引导信息,又能提供概率性结果的三阶段轨迹预测过程。首先,该方法利用无监督学习自动获取交通参与者的潜在意图集合,并应用分类网络筛选出符合当前交通参与者运动趋势的概率性目标点集合。然后,通过Transformer网络生成中间位置锚点。最后,使用连续曲线光滑连接当前位置、锚点和目标点,形成表达能力更强的概率性候选轨迹集。多个公开轨迹预测数据集的实验结果验证了该方法在提供高性能、高效率的概率性预测结果的同时,能够确保概率较高的预测结果更符合交通参与者的下一步行为。

    \item 提出了一种基于安全轨迹树网络的运动规划网络模型,减少了之前基于学习的运动规划方法在完成自动驾驶任务时出现的大量交通违规行为。针对之前方法因不能满足相关运动约束而造成的违规问题,该方法提出了一种具有曲率连续性和运动学可行性的轨迹树。该轨迹树既能够用于运动规划主任务,也能够作用于共性的轨迹预测辅助任务,从而帮助模型通过学习预测规划间的交互提升性能。针对高维栅格化特征输入可解释性差、计算效率低的问题,该方法采用包含交通参与者和局部任务路线的离散化输入表达方式,增加了模型的可解释性。该方法还利用Transformer主干网络精准提取不同输入之间的空间交互信息。针对自动驾驶汽车在复杂场景中保持长期静止不动的问题,该方法在训练过程中引入了焦点损失函数,鼓励自动驾驶车辆安全高效地完成导航任务。多个自动驾驶闭环测试基准的实验结果表明,该方法不仅在自动驾驶任务完成度和违规得分方面比之前最好的方法分别提高了39.2\%和10.6\%,而且推理速度加快了1.5倍。
\end{enumerate}

综上所述,本文所提出的单模态轨迹预测、多模态轨迹预测和运动规划方法获得了高性能的表现,具有精度高、速度快和违规驾驶行为少的优势,为保证自动驾驶安全性发挥了重要作用。

% 在主要的自动驾驶闭环测试基准的结果表明,该方法不仅获得了高性能的表现,而且比最好的方法推理速度提高了1.5倍。 在主要的行人和车辆轨迹预测公开数据集的实验结果验证了
\chinesekeywordstype{自动驾驶;轨迹预测;运动规划;自注意力模型}{应用研究}

\end{chineseabstract}

