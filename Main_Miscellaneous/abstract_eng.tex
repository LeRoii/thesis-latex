
% !TeX root = ../main.tex
\begin{englishabstract}
    % \footnotetext{*The work was supported by the Foundation (foundation ID).}
    % \astfootnote{The work was supported by the Foundation (foundation ID).}
Autonomous driving is an innovative and advanced research field in academia and industry, with potential to reduce road fatalities, improve traffic efficiency, and decrease environmental pollution. To achieve safe, reliable, stable, and efficient driving behavior, autonomous vehicles need to accurately predict the future trajectories of surrounding traffic participants and plan collision-free, kinematically feasible short-term motion trajectories. Traditional trajectory prediction methods often lack accuracy in long-term predictions, while motion planning methods based on heuristic design may lack generalization performance. In recent years, rapid advancements in data-driven deep learning methods have revolutionized prediction and planning tasks. Deep learning methods offer powerful feature fitting capabilities that enable accurate modeling of historical traffic patterns and output of future trajectories. Consequently, both prediction and planning tasks can be achieved using deep learning techniques. Despite these remarkable benefits, these methods still face various challenges, such as poor ability to deal with traffic participant heterogeneity, lack of probabilistic prediction results, and inability to guarantee trajectory smoothness. These issues pose significant safety concerns for autonomous driving and impede the further advancement of this technology.

This dissertation proposes to leverage Transformer network to address these core difficulties. The objectives of this study are twofold: 1) improving the accuracy and inference speed of practical trajectory prediction models, 2) enhancing motion planning methods to minimize traffic violations while guaranteeing the completion of autonomous driving tasks. The main contributions of this research are as follows.

\begin{enumerate}[wide,leftmargin=0pt,itemsep=\baselineskip]

    \item This dissertation proposes a new Spatio-Temporal Transformer Network for unimodal trajectory prediction, which addresses the limitations of previous homogeneous prediction methods and improves spatio-temporal interactive modeling capabilities. To address the problems of weak memory ability and unreasonable setting of spatial neighborhood range caused by the serial processing of time series data in the previous method, we adopt Transformer network and constructs a spatio-temporal graph of the whole perceptual domain. The network consists of three parts, i.e. spatio-temporal Transformer encoder, temporal Transformer encoder, and temporal Transformer decoder. The first Transformer encoder extracts spatio-temporal features by alternating between different dimensions to fully integrate spatio-temporal information. The second Transformer encoder further processes temporal information, and the temporal Transformer decoder generates unimodal trajectories for heterogeneous traffic participants. Experimental results demonstrate that the proposed method enhances key performance metrics by at least 7.2\% over state-of-the-art methods.
        
    \item This dissertation proposes a new Probabilistic Proposal Network for multimodal trajectory prediction which not only enhances the prediction accuracy of multimodal trajectory prediction, but also accelerates the inference speed. To address the problem that previous multimodal trajectory prediction methods cannot provide probabilistic prediction results, we devise a three-stage trajectory prediction process that generates target point guidance information and provides probabilistic outcomes. Firstly, the proposed method employs unsupervised learning to automatically obtain the potential intention set of traffic participants and applies a classification network to filter out a set of probabilistic target points that comply with the current movement trend of traffic participants. Next, Transformer network generates intermediate position anchors. Finally, a continuous curve is used to smoothly link the current position, anchors, and target point, producing a more expressive set of probabilistic trajectory candidates. Experimental results demonstrate that the proposed method yields high-performance and high-efficiency probabilistic prediction results while ensuring that the prediction results with higher probability align more closely with the next behavior of traffic participants.
        
    \item This dissertation proposes a new safe Trajectory Tree Network for motion planning, which can effectively reduce traffic violations while completing autonomous driving tasks. The key component of TTNet is a predefined trajectory tree that conforms to vehicle dynamics constraints and explicitly reflects different intentions. This tree is used for both the main planning task and an auxiliary trajectory prediction task. To enhance interpretability, we introduce input expressions typically used in traditional planning algorithms into our integrated framework. Additionally, to promote safe and efficient navigation, we incorporate a focal loss during training and employ a Transformer-based backbone network to accurately capture spatial interactions not only among the ego vehicle and its surroundings, but also among dynamic agents and the reference line. Experimental results demonstrate that the proposed method significantly improves task completion and violation scores by 39.2\% and 10.6\%, respectively, compared to SOTA methods while accelerating the inference speed by 1.5 times.

\end{enumerate}


In summary, our proposed methods achieve outstanding performance for unimodal trajectory prediction, multimodal trajectory prediction and motion planning, with the advantages of high precision, high speed and less driving violations, thus playing crucial roles in ensuring the safety of autonomous driving.

\englishkeywordstype{Autonomous Driving; Trajectory Prediction; Motion Planning; Transformer}{Application Research}
\end{englishabstract}