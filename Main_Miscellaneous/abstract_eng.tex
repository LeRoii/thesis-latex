
% !TeX root = ../main.tex
\begin{englishabstract}
    % \footnotetext{*The work was supported by the Foundation (foundation ID).}
    % \astfootnote{The work was supported by the Foundation (foundation ID).}
With the rapid development of the low-altitude economy, drones, as its core platform, are increasingly used in military reconnaissance, logistics distribution, public security, and other fields. Consequently, the requirements for autonomous environmental perception and intelligent capabilities of drones are continuously rising. The airborne electro-optical system is crucial for enabling drone intelligence. However, in complex and dynamic low-altitude environments, existing systems face multiple challenges. In the aspect of the algorithmic and functionality, the unique top-down perspective of drones leads to drastic variations in object scale, low pixel occupancy in images, and frequent blending with complex backgrounds, causing the performance of general-purpose detection models to degrade sharply in such scenarios. Frequent occlusions by buildings or vegetation from an aerial view can easily cause existing trackers to lose targets, making sustained long-term tracking difficult. In the aspect of system implementation, airborne platforms are subject to strict size, weight, and power (SWaP) constraints. Their core processing units must employ highly integrated embedded edge computing modules, whose computing power and memory bandwidth are orders of magnitude lower than those of server-grade GPUs. These modules can typically only support the deeply optimized lightweight convolutional neural networks. For more computationally complex advanced models based on the Transformer architecture, existing edge computing modules struggle to guarantee high frame rate real-time processing. Regarding large scale Vision-Language Models (VLMs), aside from the latest high performance computing modules, most existing devices lack the hardware foundation necessary for deploying such models. Focusing on the theme of "Multi-spectral Intelligent Electro-Optical Processing Algorithms and System Design," this dissertation, from an engineering implementation perspective, addresses the difficulties encountered in practical applications of existing electro-optical systems by providing a systematic solution ranging from core algorithms to an engineering prototype. In terms of the algorithmic, lightweight high-precision small object detection networks for visible-light and infrared images, as well as a long-term single object tracking framework with anti-occlusion capability, are designed respectively. In terms of system implementation, an integrated engineering prototype system for multi-spectral data acquisition, embedded intelligent processing, and real-time communication is designed and implemented. The specific research work is as follows.


\begin{enumerate}[wide,leftmargin=0pt,itemsep=\baselineskip]

    \item This dissertation proposes a high-performance small object detection network BAP-DETR for visible-light aerial images. Addressing the challenges of significant scale variation, complex backgrounds, and difficulty in detecting small objects in drone-based visible-light images, this work designs a Bipartite Attentive Processing Block and a Frequency-Aware Fusion Module-based Dual Fusion Feature Encoder. A channel split strategy optimizes the global modeling capabilities of convolution and self-attention. The Dual Fusion Encoder fuses high-level semantic information while preserving critical low-level features, enhancing the retention of small object features and the integration of multi-scale features. Combined with an improved loss function, the localization accuracy is improved without increasing the computational complexity of model inference. Experiments on multiple public datasets show that BAP-DETR improves the average precision(AP) by 6.9\% compared to the basline while reducing the computational load by 17.5\%, when compared to the state-of-the-art detector for UAV imagery the AP improved 3.8\%, achieving a balance between accuracy and efficiency, thus providing an effective solution for small object detection in visible-light aerial images.
        
    \item This dissertation proposes a lightweight small object detection network MFF-DCNet for drone infrared images. To address the core challenges in small object detection within drone infrared images, such as low resolution, low contrast, minimal object pixel occupancy, and limited computing power on edge devices, this dissertation designs a Depthwise Separable Cross-stage Transformer (DCFormer) module and a Multi-Feature Focus (MFF) neck structure. These designs enable real-time inference on embedded platforms while maintaining high accuracy. DCFormer combines depthwise separable convolution with a Transformer encoder to reduce computational complexity while enhancing the feature extraction capability of the backbone. The MFF neck constructs a novel feature fusion path to effectively aggregate semantic information and spatial details from different levels. Experiments on the HIT-UAV and DroneVehicle datasets show that MFF-DCNet improves detection precision (AP) by 5.8\% compared to dedicated drone image detectors, with a 10\% increase in FPS. Compared to the latest DETR and its variants, AP is improved by at least 8.2\%, and the FPS is three times that of RT-DETR. Furthermore, MFF-DCNet achieves a real-time performance of 39.6 FPS on the NVIDIA Jetson Orin NX, demonstrating its practical deployment capability in resource-constrained environments.
        
    \item This dissertation proposes an anti-occlusion long-term object tracking framework SKF-Tracker for edge computing devices. Addressing the problem of tracking failure caused by occlusion from structures (e.g., buildings, overpasses) in urban drone scenarios, this dissertation designs a modular anti-occlusion single object tracking framework that can be integrated with any base tracker. It constructs a tracking confidence assessment and occlusion detection module based on the Structural Similarity Index, combined with Kalman filter-based trajectory prediction and a dynamic template update strategy to achieve stable object tracking in complex environments. To cover diverse urban scenarios, this dissertation conducted extensive drone flight tests in multiple typical areas,such as dense urban blocks, transportation hubs, and parks to collect raw video footage. Based on this, a multi-modal UAV-perspective object tracking dataset MMUOT-1050 was constructed. This dataset comprises a total of 38 high-quality video sequences (20 visible-light and 18 infrared). Within these sequences, a total of 1,050 instances (353 in visible-light and 697 in infrared videos) were annotated with frame-by-frame bounding boxes. Each instance undergoes a complete process of being occluded and subsequently reappearing. On this dataset, SKF-Tracker achieved tracking success rates of 89.37\% for visible-light videos and 91.93\% for infrared videos, representing improvements of 14\% and 11.73\% over baseline, respectively. In terms of real-time performance, SKF-Tracker maintains high accuracy while achieving 31.25 FPS, significantly faster than deep neural network-based trackers, and does not consume NPU resources on the edge side. This provides a practical solution for stable long-term surveillance missions by drones in complex environments.
    
    \item This dissertation designs and implements an integrated multi-spectral intelligent electro-optical system prototype. It integrates multi-spectral data acquisition, embedded intelligent processing, and real-time communication. The system uses a highly integrated embedded edge computing module as its core processor, supporting the acquisition and processing of multiple visible-light and infrared video streams. In terms of software, a highly modular and extensible embedded software framework was designed. This framework employs various design patterns to achieve efficient collaboration among data acquisition, processing, and communication services. The introduction of lock-free ring buffers, memory pools, and condition variable enhances the system's real-time performance. The system establishes bidirectional data channels with a ground control station via multiple communication protocols (TCP/IP, UDP, HTTP, serial port), realizing a "human-in-the-loop" hybrid collaborative architecture. This solution was integrated and validated through field flight tests on a DJI M350 RTK drone, proving its full-chain operational capability from data acquisition and real-time processing to information interaction. This system not only provides a reliable engineering validation platform for the algorithmic development but also offers concrete technical references for subsequent algorithm deployment and system integration.
\end{enumerate}


\englishkeywordstype{Unmanned Aerial Vehicles; Electro-Optical System; Edge Computing; Object Detection and Tracking}{Application Research}
\end{englishabstract}