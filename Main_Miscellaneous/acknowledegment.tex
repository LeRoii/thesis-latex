% !TeX root = ../main.tex
% 致谢中主要感谢导师和对论文工作有直接贡献和帮助的人士和单位。
% 一般致谢的内容有:
% \begin{enumerate}[label=(\chinese*),itemindent=2em]
% 漫漫求学路,最让人回味的,莫属于读博这几年。回首初入交大时情不自禁的喜悦,经历了硕博八年洗礼后,依旧幸福感满存。谨以此文聊表感激之心。

% 衷心感谢我的导师孙宏滨教授在博士期间对我的悉心指导与关怀。在刚进组时,孙老师就将新发现号的搭建工作交与我全权负责,帮助我从系统的角度对无人驾驶整体研究有了深刻认识;在博二时,孙老师就让我担任了发现号车队队长并认真细致地指导我们准备每年的未来挑战赛,极大地提高了我的组织管理能力;在科研工作中,孙老师指点迷津,引领我做好科研探索。孙老师严谨务实的科研态度,一丝不苟的治学精神,高屋建瓴的学术见地,勤奋谦虚的个人品质都深深感染着我,激励着我,使我受益终生。

% 感谢我们敬爱的郑南宁院士。郑老师对于无人驾驶车队的关心和指导使我们整个车队的技术水平得到不断提高。感谢我博士前两年的合作导师辛景民教授的关怀,感谢魏平教授在智能车未来挑战赛备赛和比赛过程中的悉心教导,感谢王乐教授在轨迹预测方面的支持,感谢薛建儒教授、兰旭光教授、任鹏举教授、杜少毅教授、徐林海高级工程师、陈仕韬助理教授、王芳芳工程师以及其他所有人工智能学院老师在我读博期间给予的帮助和支持。

% % 
% 感谢课题组张旭翀师兄和汪航师兄对我科研工作一直以来的帮助,两位师兄扎实的理论功底和极强的解决问题能力都给我留下了很深印象。感谢沈源、张婧、刘丹为我们的学习生活提供的便利。

% 感谢王潇、史菊旺、李庚欣、陶中幸、张璞等师兄师姐在科研上的关照。感谢冯洋、杨帅、吴金强、冯超、向钊宏、陈达、张志浩、王玉学、韩伟光、权柄章、钱成龙、葛冲、陈科、李诚、罗鑫凯、陈煜炜、王申奥、李天航等师弟在发现号无人驾驶平台开发和无人车比赛中的付出。感谢戴赫、孙长峰、郑方、段景海、石刘帅等师弟在小论文上的帮助。感谢同届张剑、杨少飞、李宝婷的帮助。感谢唐浩雯师妹在科研生活中的交流与帮助。感谢好友冯立琛、丁兆伦、雷洁、马晨、荣韧闲暇时度过的快乐时光。感谢和我一起在创新港并肩战斗的赵博然,在科研和为人处世方面都对我产生了很大影响。

% 最后,感谢我的父母和家人多年来对我学习和生活上的关心和支持,是你们的坚强后盾让我能够全身心地投入到科研探索中。感恩一路有你们相伴,你们永远是我内心最温暖的港湾。


\vspace{\baselineskip}
% {\color{red} 用于双盲评审的论文,此页内容全部隐去。}

