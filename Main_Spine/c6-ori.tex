% !TeX root = ../main.tex

\xchapter{结论与展望}{Conclusion and Future Work}

\xsection{结论}{Conclusion}
轨迹预测和运动规划是自动驾驶系统的关键组成部分。轨迹预测能够为规划提供周围交通参与者未来的行动路线,运动规划完成自动驾驶汽车实时运行所需的短期无碰撞轨迹。目前针对多种场景下自动驾驶的安全性需求,轨迹预测和运动规划方法仍然存在诸多挑战。本文以解决自动驾驶预测规划任务中的关键瓶颈问题为目标,旨在提出快速高效的轨迹预测方法以及处理轨迹预测和运动规划交互的多任务联合框架,在保证预测规划质量的同时加快推理速度,从而有效提升自动驾驶的安全性。特别地,本文选择以基于Transformer网络的方法应对预测规划中所面临的算法设计难点。本文所取得的研究进展可以概括为以下三个方面。
\begin{enumerate}
    \item[(1)] 提出了基于时空Transformer网络的单模态轨迹预测方法,为简单交通场景下自动驾驶决策提供高性能的输入。单模态轨迹预测方法通常利用长短期记忆网络对轨迹数据进行串行处理,并且大多数预测方法都是针对通用的同质化交通参与者设计的,使得这些方法在处理真实的异质化交通场景时性能较差。针对这些问题,本文设计了基于Transformer网络的编码解码结构,利用由多头注意力机制和可分离卷积组成的网络层并行处理时间维度的信息,并自回归式解码出任意长度的预测轨迹。考虑到异质交通参与者的运动受到自身物理因素以及与其他交通参与者之间交互因素的双重影响,本文额外增加了交通参与者类别以及形状等输入特征信息,并构建了自动驾驶感知范围内的全域时空图模型。通过空间注意力和时间卷积的交替使用使得物理和交互两种因素得到有效考虑。实验结果表明,该单模态轨迹预测方法在真实场景中采集的自动驾驶预测数据集上的精度提高了7\%以上。
    \item[(2)] 提出了基于概率性候选轨迹网络的多模态轨迹预测方法,为复杂交通场景下的规划提供高安全性的输入。现有的大多数行人多模态轨迹预测方法不能提供概率性的预测结果,造成下游规划任务无法根据其结果评估自动驾驶汽车多种决策之间的风险权重,从而丧失了轨迹预测的实际应用价值。针对此问题,本文在自动生成的隐意图集基础上,利用分类网络输出加权性结果。为了提高预测性能,加速预测过程,该方法没有直接预测多模态的轨迹,而是采用了新的三阶段预测过程。第一阶段采用目标点分布生成、聚类和分类的三步训练策略生成目标点引导信息,并获得概率性目标点集合。第二阶段引入了方向预测策略,使用Transformer网络生成中间锚点。第三个阶段通过连续曲线形式平滑高效地连接当前位置、锚点和目标点得到最终的概率性多模态候选轨迹。实验结果表明,该方法在多个包含不同类型交通参与者的轨迹预测数据集上较之前最好的方法在性能和速度都有提升。
    \item[(3)] 提出了基于安全轨迹树网络的运动规划方法,为完成自动驾驶规划任务提供新的设计框架。当前主流的基于深度学习网络的规划方法是先将传感器输入转换为高维栅格化特征然后再映射成运动规划轨迹的端到端方法。这些方法主要侧重于增强对高维栅格化特征的提取能力,忽略了关键的任务输出部分,导致输出轨迹的运动学约束无法保证,造成其完成自动驾驶任务时会出现大量交通违规行为。针对这些问题,该方法摒弃了从高维栅格化特征到运动规划轨迹的思路,利用传统规划方法中包含交通参与者边界框和局部参考线的离散化表达方式,设计了可以无缝接入大部分自动驾驶整体框架的基于学习的规划方法。该方法还提出了轨迹树的概念,可以为所有交通参与者提供大量具有曲率连续性和运动学可行性的候选叶轨迹模板。该方法采用了多任务学习框架,将离散化表达方式输入到空间Transformer主干网络后,不仅可以得到运动规划轨迹,而且可以输出周围交通参与者的多模态预测轨迹。在自动驾驶仿真闭环测试中的结果表明,该方法在路径完成度、交通违规得分和综合驾驶得分上都取得了最好的结果。
\end{enumerate}

本文在对自动驾驶轨迹预测和运动规划任务充分探索的基础上,从预测精度、推理速度、驾驶性能等各类指标综合考虑,深入研究了单模态和多模态轨迹预测以及运动规划方法特点和技术需求。基于Transformer网络对于时空信息的高效处理能力,本文提出了一系列兼顾精度和实时性能的高效轨迹预测方法,并进一步地将预测任务与运动规划任务相结合,设计了轨迹预测和运动规划的自动驾驶多任务解决方案,具有重要的研究与应用价值。


\xsection{展望}{Future Work}
本文主要研究的是面向安全自动驾驶的轨迹预测与运动规划。虽然本文所提出的方法可以实现异质交通参与者的单模态和多模态轨迹预测,但是相比于结合高精度地图信息的轨迹预测在精度上和实用性上仍然存在差距。同时,实现的运动规划框架虽然驾驶得分等指标已经达到当前最好的性能,但是仍然无法满足复杂自动驾驶测试场景的需求。之后的研究工作将继续在现有的研究基础上展开,进一步探索基于深度学习的自动驾驶轨迹预测与运动规划方法的实用化,并期望通过算法研究与实际自动驾驶研究平台算法部署两个分支路线推动整体自动驾驶技术的发展。其中,可能的探索方向包括:

\paragraph{基于深度聚类的多模态轨迹预测方法研究}
本文所提出的多模态轨迹预测方法研究使用了以“k-means”为基础的无监督学习方法尽可能精准地收集交通参与者的隐式运动模式。但是“k-means”聚类方法需要预先设置聚类的类别数量,而以“DBSCAN”为核心的其他传统聚类方法虽然不用设置类比数量的超参数,但是其针对交通参与者生成的隐特征聚类质量明显劣于“k-means”聚类方法。近年来,大量的深度聚类方法涌现出来。其核心思想是通过表示学习和聚类之间相互提升,良好的表示改进聚类的质量,而良好的聚类会为表示学习提供良好的监督信号。因为深度聚类方法不仅对数据分布的假设更少,而且通常能够产生更好的无监督学习结果,所以可以将需要超参数设置的传统“k-means”聚类方法转变为深度聚类方法,从而自动从轨迹预测数据集大量的数据中提取到丰富的运动行为模式。采用深度聚类方法的另一个优势是可以和多模态轨迹预测网络联合训练,减少生成过程的中间步骤,并通过预测误差的后向传播的方式进一步提升聚类质量。

\paragraph{基于高精度地图信息的轨迹预测方法研究}
本文所提出的方法考虑了目标指导信息的利用、交通参与者自身的时间交互和交通参与者之间的空间交互,忽略了交通参与者与场景之间的交互特征。考虑交通参与者与场景之间的交互特征可以为轨迹预测方法设计引入更多的信息。在搭载激光雷达和相机的实际自动驾驶系统中,利用感知模块和定位建图模块能够构建出包含车道线、停止线、斑马线、红绿灯及其他交通标识的高精度语义地图。在规划模块中,包含丰富语义信息的局部参考线就是根据高精度语义地图生成的。自动驾驶汽车能够根据局部参考线提供的地图信息及周围动静态目标信息精准还原出其当下所处的真实环境。这种规划模块的处理思路完全可以引入针对自动驾驶汽车周围动态交通参与者的轨迹预测中。在进行轨迹预测时,可以根据高精度地图提供丰富语义信息,构建出感知范围内所有交通参与者其自身所处的不确定性环境。后续工作将考虑以栅格地图或者矢量化地图的形式引入高精度地图语义信息,并采用合适的网络模型进行不确定性估计和特征提取。

\paragraph{强化学习和模仿学习相结合的自主决策规划算法研究与应用}
本文提出的轨迹预测与规划一体化框架能够完成CARLA仿真器闭环测试环境下的多种挑战性任务路线。但是,通过空间Transformer编码器产生的决策部分仍然非常简单,只能够完成跟车以及固定地点的汇入车流和换道等简单决策,并不能够做出自由换道等高级决策。深度强化学习能够有效应对自动驾驶多种复杂场景下的决策。然而,当前深度强化学习的方法同样存在样本复杂性高和稳定性弱的问题。具有专家知识的模仿学习方法可以弥补上述缺陷。因此后续工作考虑将来自强化学习探索方面的优势与模仿学习在学习专家策略方面的能力相结合,寻求采用易于实现的通用强化模仿学习方法完成自动驾驶自主决策规划任务。模仿学习训练得到的专家策略能够为深度强化学习提供性能优越的初始化智能体,而深度强化学习通过深入探索,可以突破专家策略的局限,如专家选择的低效保守的跟车策略等。这种强化学习和模仿学习相结合的方式对于进一步提升甚至是跨越式提升基于学习的规划方法在闭环测试中的性能具有重要研究价值。

% 西安交通大学“发现号”
\paragraph{多模态轨迹预测及运动规划方法在实际自动驾驶平台的部署}
本文所提出的多模态轨迹预测方法仍然只是在数据集上进行了验证。未来工作考虑将其部署在实车平台上,真正从实际应用的角度去发现和解决多模态轨迹预测问题。同时,现有的大部分实际自动驾驶平台中使用的规划方法仍然无法很好地处理多模态的轨迹预测输入。如百度Apollo开源版本的规划方法只选择了概率最大的多模态轨迹预测结果作为规划输入,相当于转化为了单模态的轨迹预测输入。因此,如何更好地处理多模态轨迹预测输入对于实际自动驾驶汽车具有重要意义。另一方面,本文所提出的规划预测一体化方法仍然停留在仿真测试阶段,其选择的专家数据集也是从CARLA仿真器中采集的。打破模仿学习和强化学习方法从仿真环境到现实环境瓶颈的“sim-to-real”研究方向是当前基于学习的规划方法落地的重要保证之一。未来工作拟结合之前在实车平台积累的研发优势,为基于学习的规划方法突破实用化限制开展相关研究工作。

