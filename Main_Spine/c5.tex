% !TeX root = ../main.tex

\xchapter{智能光电系统实现与跟踪算法集成验证}{}
本章将详细介绍基于软硬件协同设计理念的智能光电系统整体实现方案,具体涵盖多光谱传感器采集、基于Nvidia Jetson和RK3588边缘计算平台的算法部署与优化,以及采用模块化设计的端侧全功能软件框架。为了确保系统中各模块间高效、可靠的数据交换与控制,本章专门设计了配套的轻量级通信协议,定义了传感器、计算单元与上位机之间的标准化数据接口。同时,开发了功能完善的上位机软件,集成了实时状态监控、算法参数调整、结果可视化以及视频与日志记录功能,为系统调试、性能评估和人机交互提供了直观友好的操作界面。其次,针对动态复杂环境中目标易丢失的难题,本章提出了面向边缘计算设备的融合重识别机制与自适应模板更新的抗遮挡长时目标跟踪算法,并将其集成到智能光电系统中,通过实际场景测试验证了其鲁棒性。本章工作不仅验证了算法的工程可行性,也为构建自主智能光电系统提供了从底层硬件、核心算法到交互软件的完整原型参考与实践依据。

\xsection{引言}{Introduction}
随着低空经济的蓬勃发展和无人机任务复杂度的不断提升,机载光电系统正经历着从“被动观测”到“主动感知”、从“功能单一”到“多任务智能”的全面角色转变。在信息化、网络化、智能化融合发展的背景下,现实环境呈现出高度动态、高度复杂、信息饱和的特征,对环境感知的实时性、准确性、自主性提出了更严格的要求。这一系列需求驱动机载光电系统向智能化方向发展。智能化是指系统具备从原始数据中自主提取信息、在复杂不确定环境下进行稳定推理、并依据任务目标做出及时响应或决策的闭环能力,最终目标是使光电系统成为具备一定“认知”能力的空中智能体,能够在最小化人工干预的情况下,独立完成从搜索、发现、识别、跟踪到评估的完整环节。然而,现有系统距离实现上述理想的、完整的“智能化”,仍存在显著的差距,主要体现在硬件与软件两个层面。

在硬件层面,机载平台,尤其是中小型无人机,对载荷有着严格的尺寸、重量、功耗和成本限制,这导致机载信息处理单元算力与内存带宽有限,与云服务器或大型地面站相比,存在数量级差距。这种有限的计算资源与当前先进的视觉算法对算力的庞大需求构成了直接矛盾,尤其对于参数量庞大、计算密集的Transformer模型。许多在实验室标准数据集上表现优异的算法,因其复杂的计算度和巨大的延迟,难以直接部署到机载平台进行实时推理。因此,现有的“智能”往往是一种被严重约束的智能,需要在算法性能与实时性之间做出艰难平衡,导致许多先进的感知能力在端侧无法充分发挥。另外,更深层次的制约源于机载环境严苛的功耗与热管理约束。即便在机载平台中采用了具有高算力的边缘计算模块,其性能也往往受到物理规律的严峻挑战。机载平台的电能供给极为有限。无人机续航直接依赖于机载电池,为所有载荷(包括飞控、传感器、通信与计算单元)分配固定的功率预算。一个在实验室中峰值功耗可达数十瓦的高性能计算模块,在机载环境下可能因超出功率预算而无法被允许持续运行在峰值状态。设计者通常被迫在其峰值性能和平均功耗之间做出权衡,通过软件或硬件手段设置功耗墙,这意味着硬件在其工作的大部分时间里都无法以其最大标称频率运行。其次,与功耗紧密相关的热管理问题在机载环境中尤为突出。计算芯片在高负载下产生的热量必须被有效散出,以防止芯片结温超过安全阈值导致降频、重启甚至永久损坏。然而,无人机紧凑的机体内通常缺乏空间部署大型主动散热系统(如风扇阵列或液冷回路),主要依赖有限表面积下的被动散热或风冷。在低空低速或悬停状态下,对流散热效率降低,极易导致热量积聚。因此,即使硬件模块在理论上具备高算力,在实际飞行中,其持续运算往往会迅速触发温度保护机制,迫使系统动态降频以控制温度,使得实际可持续的运算性能远低于标称峰值。这种由功耗与热约束导致的性能降级对于实际工程应用具有重大影响。许多在实验室标准数据集上、有充足散热保障的硬件平台上表现优异的复杂算法,一旦置于真实的机载环境,便会因计算单元的瞬时算力下降和频繁降频,而遭遇难以预测的性能波动和延迟激增。算法原本稳定的处理流水线被破坏,实时性指标急剧恶化。因此,现有的机载“智能”往往是一种被双重约束的智能:一是底层硬件资源的绝对不足,二是即使存在的硬件资源也因物理限制而无法全力输出。系统设计必须在算法理论性能、实时性要求、功耗上限与散热能力之间进行多维度的、艰难的平衡,这导致许多先进的感知模型在端侧无法充分发挥其潜力,甚至不得不为保障最基本的运行可靠性而牺牲精度与功能。

在软件与算法层面,现有系统更多地解决了“有无”问题,但与“智能”的本质要求尚有巨大差距。现有功能大多围绕单一、特定的任务(如“对指定类别的目标进行检测或跟踪”)进行优化。其算法流程往往是链式或并行的,检测、跟踪等模块相对独立,信息流动单向。算法模型通常在离线状态下用固定数据集训练完成,一旦部署,其参数和结构就已经固化。面对训练数据未充分涵盖的新环境、新目标类型或新型干扰,系统性能会急剧下降。
在特定数据集上取得高精度的算法,在实际复杂的使用环境中,其性能边界往往模糊不清。对于采用深度神经网络的算法,当其在边缘设备部署时,必须借助边缘芯片厂商提供的专用工具链(如NVIDIA的TensorRT、华为的CANN、瑞芯微RKNN等)对模型进行量化、编译与优化,以使其能在特定的异构计算硬件上高效执行。然而,这个过程,往往以牺牲一定的算法精度为代价,导致在标准数据集上评测出的高精度无法无损地转化到实际应用中。从FP32到FP16或INT8的量化,使可表示的数值分辨率急剧下降。对于特征图中表征微弱信号或精细结构的小数值,量化后可能归零或产生较大相对误差,直接影响对小目标、低对比度目标或目标边缘的感知精度。这些工具链为了极致性能,会将多个基础算子(如卷积、批归一化、激活)融合为一个复合算子,并进行内存访问优化。然而,这种融合有时会改变运算的微小数值顺序或精度,可能与原始训练时定义的数学行为存在细微偏差,在网络中被逐层放大,最终导致误检、漏检或跟踪漂移。因此,一个神经网络经过量化部署到机载边缘平台后,其实际性能存在不可避免的损失,这种损失使得算法在实际复杂环境下的性能边界更加模糊和脆弱。

本章的核心将从纯算法研究与理论分析,转向以实际部署与应用验证为导向的系统工程实践。具体而言,我们首先构建一套完整的软硬件协同设计架构,涵盖从多光谱传感器数据采集、到基于边缘计算设备的算法深度优化部署、再到模块化软件框架实现的全链路集成方案。进而,针对复杂动态场景中最具挑战性的目标持续跟踪问题,本章将提出并集成一种创新的抗遮挡长时跟踪算法,并通过实际场景测试,验证评估其在接近真实条件中的鲁棒性与实用性。最终,本章旨在打通从先进算法到可靠系统产品的关键路径,为智能光电系统的工程化与实战化提供一套经过验证的方法论与实例。



\xsection{智能光电系统整体实现方案}{Overall Implementation Scheme of Intelligent Electro-Optical System}

机载光电系统主要由红外分系统、可见光分系统、激光分系统、稳定跟踪分系统、系统控制分系统、图像处理分系统、和结构分系统等组成,通过各分系统之间的协调工作完成系统的功能和性能。其工作原理如图\ref{sys-arch}所示。

\begin{figure}[H]
\centering
\includegraphics[width=\textwidth]{sys-arch.png}
\caption{机载光电系统组成框图}
\label{sys-arch}
\end{figure}





智能光电系统整体实现方案包括多光谱传感器采集、基于Nvidia Jetson和RK3588边缘计算平台的算法部署与优化,以及采用模块化设计的端侧全功能软件框架。

    传感器数据采集
    边缘计算核心模组

\xsection{面向边缘计算设备的抗遮挡长时跟踪算法}{}

\xsection{实验结果与分析}{}


\xsubsection{基于Nvidia Jetson和RK3588边缘计算平台的算法部署与优化}{Algorithm Deployment and Optimization on Nvidia Jetson and RK3588 Edge Computing Platform}
基于Nvidia Jetson和RK3588边缘计算平台的算法部署与优化是智能光电系统的核心,包括目标检测算法和目标跟踪算法等。其中,目标检测算法主要用于检测目标,目标跟踪算法主要用于跟踪目标。

\xsubsection{采用模块化设计的端侧全功能软件框架}{Modular Design of End-to-end Full-function Software Framework}