% !TeX root = ../main.tex

\xchapter{智能光电系统实现与跟踪算法集成验证}{}
本章将详细介绍基于软硬件协同设计理念的智能光电系统整体实现方案,具体涵盖多光谱传感器采集、基于Nvidia Jetson和RK3588边缘计算平台的算法部署与优化,以及采用模块化设计的端侧全功能软件框架。为了确保系统中各模块间高效、可靠的数据交换与控制,本章专门设计了配套的轻量级通信协议,定义了传感器、计算单元与上位机之间的标准化数据接口。同时,开发了功能完善的上位机软件,集成了实时状态监控、算法参数调整、结果可视化以及视频与日志记录功能,为系统调试、性能评估和人机交互提供了直观友好的操作界面。其次,针对动态复杂环境中目标易丢失的难题,本章提出了面向边缘计算设备的融合重识别机制与自适应模板更新的抗遮挡长时目标跟踪算法,并将其集成到智能光电系统中,通过实际场景测试验证了其鲁棒性。本章工作不仅验证了算法的工程可行性,也为构建自主智能光电系统提供了从底层硬件、核心算法到交互软件的完整原型参考与实践依据。

\xsection{引言}{Introduction}
随着低空经济的蓬勃发展和无人机任务复杂度的不断提升,机载光电系统正经历着从“被动观测”到“主动感知”、从“功能单一”到“多任务智能”的全面角色转变。在信息化、网络化、智能化融合发展的背景下,现实环境呈现出高度动态、高度复杂、信息饱和的特征,对环境感知的实时性、准确性、自主性提出了更严格的要求。这一系列需求驱动机载光电系统向智能化方向发展。智能化是指系统具备从原始数据中自主提取信息、在复杂不确定环境下进行稳定推理、并依据任务目标做出及时响应或决策的闭环能力,最终目标是使光电系统成为具备一定“认知”能力的空中智能体,能够在最小化人工干预的情况下,独立完成从搜索、发现、识别、跟踪到评估的完整环节。然而,现有系统距离实现上述理想的、完整的“智能化”,仍存在显著的差距,主要体现在硬件与软件两个层面。

在硬件层面,机载平台,尤其是中小型无人机,对载荷有着严格的尺寸、重量、功耗和成本限制,这导致机载信息处理单元算力与内存带宽有限,与云服务器或大型地面站相比,存在数量级差距。这种有限的计算资源与当前先进的视觉算法对算力的庞大需求构成了直接矛盾,尤其对于参数量庞大、计算密集的Transformer模型。许多在实验室标准数据集上表现优异的算法,因其复杂的计算度和巨大的延迟,难以直接部署到机载平台进行实时推理。因此,现有的“智能”往往是一种被严重约束的智能,需要在算法性能与实时性之间做出艰难平衡,导致许多先进的感知能力在端侧无法充分发挥。另外,更深层次的制约源于机载环境严苛的功耗与热管理约束。即便在机载平台中采用了具有高算力的边缘计算模块,其性能也往往受到物理规律的严峻挑战。机载平台的电能供给极为有限。无人机续航直接依赖于机载电池,为所有载荷(包括飞控、传感器、通信与计算单元)分配固定的功率预算。一个在实验室中峰值功耗可达数十瓦的高性能计算模块,在机载环境下可能因超出功率预算而无法被允许持续运行在峰值状态。设计者通常被迫在其峰值性能和平均功耗之间做出权衡,通过软件或硬件手段设置功耗墙,这意味着硬件在其工作的大部分时间里都无法以其最大标称频率运行。其次,与功耗紧密相关的热管理问题在机载环境中尤为突出。计算芯片在高负载下产生的热量必须被有效散出,以防止芯片结温超过安全阈值导致降频、重启甚至永久损坏。然而,无人机紧凑的机体内通常缺乏空间部署大型主动散热系统(如风扇阵列或液冷回路),主要依赖有限表面积下的被动散热或风冷。在低空低速或悬停状态下,对流散热效率降低,极易导致热量积聚。因此,即使硬件模块在理论上具备高算力,在实际飞行中,其持续运算往往会迅速触发温度保护机制,迫使系统动态降频以控制温度,使得实际可持续的运算性能远低于标称峰值。这种由功耗与热约束导致的性能降级对于实际工程应用具有重大影响。许多在实验室标准数据集上、有充足散热保障的硬件平台上表现优异的复杂算法,一旦置于真实的机载环境,便会因计算单元的瞬时算力下降和频繁降频,而遭遇难以预测的性能波动和延迟激增。算法原本稳定的处理流水线被破坏,实时性指标急剧恶化。因此,现有的机载“智能”往往是一种被双重约束的智能:一是底层硬件资源的绝对不足,二是即使存在的硬件资源也因物理限制而无法全力输出。系统设计必须在算法理论性能、实时性要求、功耗上限与散热能力之间进行多维度的、艰难的平衡,这导致许多先进的感知模型在端侧无法充分发挥其潜力,甚至不得不为保障最基本的运行可靠性而牺牲精度与功能。

在软件与算法层面,现有系统更多地解决了“有无”问题,但与“智能”的本质要求尚有巨大差距。现有功能大多围绕单一、特定的任务(如“对指定类别的目标进行检测或跟踪”)进行优化。其算法流程往往是链式或并行的,检测、跟踪等模块相对独立,信息流动单向。算法模型通常在离线状态下用固定数据集训练完成,一旦部署,其参数和结构就已经固化。面对训练数据未充分涵盖的新环境、新目标类型或新型干扰,系统性能会急剧下降。
在特定数据集上取得高精度的算法,在实际复杂的使用环境中,其性能边界往往模糊不清。对于采用深度神经网络的算法,当其在边缘设备部署时,必须借助边缘芯片厂商提供的专用工具链(如NVIDIA的TensorRT、华为的CANN、瑞芯微RKNN等)对模型进行量化、编译与优化,以使其能在特定的异构计算硬件上高效执行。然而,这个过程,往往以牺牲一定的算法精度为代价,导致在标准数据集上评测出的高精度无法无损地转化到实际应用中。从FP32到FP16或INT8的量化,使可表示的数值分辨率急剧下降。对于特征图中表征微弱信号或精细结构的小数值,量化后可能归零或产生较大相对误差,直接影响对小目标、低对比度目标或目标边缘的感知精度。这些工具链为了极致性能,会将多个基础算子(如卷积、批归一化、激活)融合为一个复合算子,并进行内存访问优化。然而,这种融合有时会改变运算的微小数值顺序或精度,可能与原始训练时定义的数学行为存在细微偏差,在网络中被逐层放大,最终导致误检、漏检或跟踪漂移。因此,一个神经网络经过量化部署到机载边缘平台后,其实际性能存在不可避免的损失,这种损失使得算法在实际复杂环境下的性能边界更加模糊和脆弱。

本章的核心将从纯算法研究与理论分析,转向以实际部署与应用验证为导向的系统工程实践。具体而言,我们首先构建一套完整的软硬件协同设计架构,涵盖从多光谱传感器数据采集、到基于边缘计算设备的算法深度优化部署、再到模块化软件框架实现的全链路集成方案。进而,针对复杂动态场景中最具挑战性的目标持续跟踪问题,本章将提出并集成一种创新的抗遮挡长时跟踪算法,并通过实际场景测试,验证评估其在接近真实条件中的鲁棒性与实用性。最终,本章旨在打通从先进算法到可靠系统产品的关键路径,为智能光电系统的工程化与实战化提供一套经过验证的方法论与实例。



\xsection{智能光电系统整体实现方案}{Overall Implementation Scheme of Intelligent Electro-Optical System}

\xsubsection{系统组成及工作原理}{}
机载光电系统主要由红外分系统、可见光分系统、激光分系统、稳定跟踪分系统、系统控制分系统、图像处理分系统、和结构分系统等组成,其工作原理如图\ref{sys-arch}所示。从系统运行角度来说,各分系统是一个有机的整体,通过各分系统之间的协调工作实现相应的功能性能。从物理结构角度上说,光电系统有的分系统与其他分系统综合为一体,如红外分系统、可见光分系统等光机部件交织在一起,有的分系统是模块化。

\begin{figure}[H]
\centering
\includegraphics[width=\textwidth]{sys-arch.png}
\caption{机载光电系统组成框图}
\label{sys-arch}
\end{figure}

\subsubsection{红外分系统}
红外分系统是机载光电系统实现全天时全天候环境感知能力的主要传感器,由于它利用的是物体(目标/背景)自身发射的红外谱段的光线,可以完全不依赖周边环境的光源,是短波、微光以及激光等不可替代的光学探测通道,通常是光电系统的必备传感器。红外分系统主要由红外光学镜头、支撑光学镜组的精密光机结构、红外探测器、信息处理部件、视场切换机构、变焦机构等部件组成,如图\ref{ircam-sys}所示。红外分系统主要功能是对地面景物热辐射信息进行光电转换,输出视频信号给信息处理单元用于后续检测跟踪任务,同时为了适应不同环境下得到便于处理分析的图像,红外分系统具有电子变焦、黑白热切换、图像冻结、增益控制、偏置控制等功能。一般情况下,红外分系统在光电系统中是一个独立的组件,能够方便地进行更换,称为红外热像仪。有时红外光学光机部分与可见光光学、激光光学融合在一起形成一个十分复杂的光机构型,如共光路构型,瞄准吊舱多采用此种构型。

\begin{figure}[H]
\centering
\includegraphics[width=\textwidth]{ircam-sys.png}
\caption{红外分系统组成}
\label{ircam-sys}
\end{figure}

红外分系统在设计过程中主要考虑以下几个方面。
\paragraph{波段选择}
红外系统的波长范围确定非常重要,同时工作波长也是系统总体指标之一。在红外分系统设计中,根据大气窗口可以选择中波(3-5μm)和长波(8-12μm)两个波段,具体波段的选择主要与目标和背景的辐射特性、大气传输特性和系统性能有关。设目标的辐射系数为$\varepsilon(\lambda,T)$,目标的积分辐射出射度为:
\begin{equation}
    M\left(\lambda_1-\lambda_2\right)=\int_{\lambda_1}^{\lambda_2} \varepsilon(\lambda, T) M_{\mathrm{eo}}(\lambda, T) \mathrm{d} \lambda
\end{equation}

根据维恩位移定律
\begin{equation}
    \lambda_mT=2897.8μm\cdot K
\end{equation}

由于目标辐射峰值波长与温度成反比,从辐射特性的角度分析,对于目标特征为高温物体的红外系统选用中波波段,对于目标特征为常温物体的红外系统选用长波波段。但红外系统波段的选择还要考虑使用环境,由于空气湿度对长波影响较对中波影响严重,对于海上高温高湿的环境一般选用中波红外系统。从实际应用角度,由于近几年中波红外制冷型探测器灵敏度得到了很大程度的提高,工艺稳定,同时制冷型长波红外探测器较中波红外探测器存在成本高、稳定性较差等问题,因此目前国外研制的光电跟瞄系统采用中波红外的比例很高,在具体设计中红外分系统的波段选择要根据实际系统的使用情况综合考虑。

\paragraph{光学设计}
红外光学镜头是保证系统光学性能的关键部件,在很大程度上影响着红外分系统的性能。在红外探测器确定后,系统总体设计中关于红外发现和识别距离的指标主要依靠光学设计保证。光学设计需要综合考虑视场大小、视场数量、焦距和传递函数等指标的要求,同时也要满足尺寸空间、重量、温度振动等环境条件的约束。

\paragraph{视场变换}
红外分系统一般设计两个以上光学视场,宽视场用于广域搜索潜在目标,窄视场用于分辨识别目标的轮廓细节。视场切换的实现一般采用两种光学构型,第一种采用轴向移动透镜实现视场切换,第二种采用切入/切出变倍镜组实现视场切换。

\paragraph{图像增强}
由于红外图像动态范围较大,在将其转换为适合人眼观察的模拟图像过程中,容易造成图像细节的缺失,影响人眼的观察效果。红外探测器输出的原始数据一般为16位,而显示器显示的图像数据一般为8位。现代红外相机通常在内部集成了一系列基础的图像增强与校正功能,优化输出图像的视觉质量,其中以非均匀性校正和坏点补偿为基础,消除由探测器像元响应差异引起的固定模式噪声,确保图像均匀性,同时,对传感器缺陷造成的图像中的亮点或暗点进行检测和补偿,随后,通过基于查找表的动态范围压缩与对比度增强算法,将宽广的原始灰度范围进行智能映射,有选择地拉伸感兴趣温区的对比度以凸显细节,抑制非关键区域的灰度变化。同时,为优化主观视觉效果,相机常提供多种伪彩色编码方案,将灰度温度信息转换为彩色图像,以增强人眼对不同温度分布的辨识能力。此外,实时噪声抑制与电子稳像等辅助功能也常被集成,以进一步提升输出图像的清晰度与稳定性。这一系列嵌入式的处理步骤均在数据输出前完成,为后端分析系统提供实时、清晰、细节丰富的可视化图像。


\subsubsection{可见光分系统}
可见光探测在机载光电系统中与红外探测一样承担着系统总体的性能指标要求,可见光成像系统获得的图像与人眼观察到的图像一样,非常有利于人的视觉系统识别目标,在光电系统中是必备的传感器。虽然红外探测白天也可以使用,但不能替代可见光探测,两者各有优势。可见光分系统有时是一个独立的组件,常被称为电视摄像机,有时同红外、激光和微光等综合设计在一起。

可见光分系统由光学镜头、调光机构、变倍机构、CCD探测器和成像处理模块等组成,如图所示。主要功能是收集来自目标/景物的可见光信息,通过光电转换和信号处理,输出标准视频信号用于后续处理。变视场控制机构通过控制透镜组的移动完成视场的切换,调光控制机构通过电子快门和光圈使CCD实现在全照度范围的清晰成像。

\begin{figure}[H]
\centering
\includegraphics[width=\textwidth]{viscam-sys.png}
\caption{可见光分系统组成}
\label{viscam-sys}
\end{figure}

可见光分系统光学镜组的设计与红外光学设计要考虑的问题大致相同,设计方法相似。可见光分系统有时采用单色成像,有时采用彩色成像。目前彩色成像CCD非常成熟,在光电系统中应用较为普遍。CCD探测器由高感光度的半导体材料制成,包含有众多的感光元件,每个感光元件对应一个像素,通过像素敏感光线最终构成一幅完整的图像。随着技术的进步,CCD探测器的分辨率得到了大幅度的提高,目前2k以上的产品已在机载光电系统中得到了应用,这将有效提高可见光分系统的空间分辨率和探测识别距离,也是可见光探测存在的一个必要和优势。

\subsubsection{系统控制与图像处理分系统}
在传统的光电系统架构中,系统控制分系统与图像处理分系统在功能与硬件上通常是分离的。图像处理分系统作为数据处理的核心,主要负责接收来自可见光、红外等传感器的原始数据流,并执行从基础图像预处理到高级智能分析等一系列计算密集型任务。系统控制分系统负责与飞行控制单元、稳定平台、任务载荷管理器等外部模块进行实时通信,解析并执行来自上位机的任务指令,同时监控各个子系统的状态,并进行统一的调度与协同管理。这两个分系统之间通过高速总线进行数据与指令交互。

随着嵌入式硬件技术的发展,尤其是高性能片上系统与异构计算架构的成熟,上述分离式设计正被高度集成的方案所取代。当前的主流趋势是将这两个分系统的功能,深度融合到单一的边缘计算模组上。如图\ref{imgproc-sys}所示,此类模组(如瑞芯微RK3588、NVIDIA Jetson系列)通常采用“CPU+GPU+NPU”的异构设计:CPU 核心负责运行轻量化的操作系统和系统控制分系统的所有逻辑,处理通信、调度与状态管理等任务,GPU 和专用的神经处理单元(NPU) 则负责加速图像处理分系统中的深度学习算法,实现实时的目标识别与跟踪。FPGA常被集成或作为协处理器,用于处理传感器数据接入、预处理等低延迟的流水线操作。

\begin{figure}[H]
\centering
\includegraphics[width=\textwidth]{imgproc-sys.png}
\caption{系统控制与图像处理分系统组成}
\label{imgproc-sys}
\end{figure}

这种硬件层面的集成通过消除分系统间的物理接口和总线传输,极大地降低了系统内部通信延迟,使得控制指令能够更快速地响应图像处理结果,另一方面,硬件资源的统一管理提升了整体效率,并大幅减少了系统的体积、重量和功耗,这对于空间和能源受限的机载平台至关重要。因此,现代智能光电系统的设计,已成为在单一高性能计算平台上,对实时数据流处理、外部交互指令响应及内部系统控制多任务协同优化的系统工程。

\subsubsection{稳定跟踪分系统}
稳定跟踪分系统的主要功能是利用陀螺构成相对惯性空间稳定的陀螺稳定平台,隔离载机飞行中产生的姿态变化、振动等扰动,为光学传感器提供一个在惯性空间中高度稳定的物理指向基准,并在图像处理分系统的配合下,驱动稳定平台完成对目标的精确跟踪。对于大多数的机载光电系统,为了实现较高的精度要求,其稳定跟踪分系统的组成都是比较复杂的,但是从控制的角度来看,一般是一个或多个伺服控制系统的结合。其组成如图\ref{stable-sys}所示。

\begin{figure}[H]
\centering
\includegraphics[width=\textwidth]{stable-sys.png}
\caption{稳定跟踪分系统组成}
\label{stable-sys}
\end{figure}

稳定跟踪分系统是典型的高度集成的机电伺服单元,其硬件架构围绕稳定平台结构这一核心机械骨架构建。该平台通常采用两轴(方位、俯仰)或三轴(增加横滚)框架设计,由高刚度、轻量化的材料精密加工而成,直接承载并固定光电传感器载荷,为其提供物理安装基准和运动自由度。驱动组件负责执行控制指令,输出动力。惯性基准单元,通常为光纤或微机电陀螺,其核心功能是测量平台框架相对于惯性空间的角运动,建立稳定控制的绝对基准。控制线路集成了运行核心控制算法的处理器、指令输出接口、电机功率放大器及其他状态控制电路,负责处理传感器信号、解算控制参数并驱动电机。角度传感器用于测量框架之间以及框架与载机基座之间的相对角位置。

稳定回路是整个分系统的基础闭环,其核心任务是隔离扰动,保持光电传感器瞄准线的稳定。当载机机动或受外界扰动时,扰动力矩会作用于平台框架,平台框架相对于此时陀螺的惯性基准产生了运动。陀螺测量到这一微小的角速度或位移。该信号经过滤波、解调与放大后,被送入数字控制器,解算出平台偏离预定惯性基准的偏差量,随后,控制器根据此偏差,计算出补偿控制指令。该指令通过功率驱动电路放大,驱动框架轴上的力矩电机产生补偿力矩。这个补偿力矩驱动平台框架朝相反方向运动,从而抵消了外部扰动的影响,使光电传感器的视轴保持动态稳定。

跟踪回路是构建于稳定回路之上的控制闭环,其核心功能是驱动已稳定的视轴,主动、持续地跟随场景中的运动目标。当目标与载机存在相对运动时,其在传感器成像平面上的位置会偏离图像中心。图像跟踪器(通常为图像处理分系统中的专用算法)对视频流进行实时处理,计算出目标相对于图像中心的二维像素偏差,即脱靶量。脱靶量被传输至跟踪控制器,控制器结合当前传感器的焦距、平台姿态等参数,将其转换为使目标重新回归图像中心所需的平台进动角速度或角位移指令。此指令作为新的期望输入,被叠加到稳定回路的控制环路中。在保持自身稳定性的同时,稳定平台在驱动电机的带动下开始执行受控的角运动,驱动视轴朝着目标偏移的方向运动。随着视轴的主动调整,目标在图像中的位置被逐步拉回中心。图像跟踪器持续检测并返回脱靶量,形成一个闭合的视觉伺服闭环。



智能光电系统整体实现方案包括多光谱传感器采集、基于Nvidia Jetson和RK3588边缘计算平台的算法部署与优化,以及采用模块化设计的端侧全功能软件框架。

    传感器数据采集
    边缘计算核心模组

\xsection{面向边缘计算设备的抗遮挡长时跟踪算法}{}

\xsection{实验结果与分析}{}


\xsubsection{基于Nvidia Jetson和RK3588边缘计算平台的算法部署与优化}{Algorithm Deployment and Optimization on Nvidia Jetson and RK3588 Edge Computing Platform}
基于Nvidia Jetson和RK3588边缘计算平台的算法部署与优化是智能光电系统的核心,包括目标检测算法和目标跟踪算法等。其中,目标检测算法主要用于检测目标,目标跟踪算法主要用于跟踪目标。

\xsubsection{采用模块化设计的端侧全功能软件框架}{Modular Design of End-to-end Full-function Software Framework}