% !TeX root = ../main.tex

\xchapter{基于安全轨迹树网络的运动规划}{Motion Planning using Safe Trajectory Tree Network for Autonomous Driving}

由于传统的自动驾驶运动规划方法需要大量人工设计的复杂规则,当前研究转向了以神经网络为核心的更加简单有效的运动规划方法。然而,现有基于神经网络的运动规划方法在完成自动驾驶任务时会出现许多交通违规问题,导致任务的完成率低,综合性能指标差。此外,这些运动规划方法使用的高维栅格化特征输入存在计算效率低下的问题。本章研究主要以基于模仿学习的运动规划方法展开,解决如何在高质量的完成自动驾驶任务的同时尽可能地减少交通违规问题。为此,本章设计了平滑且运动学可行的安全轨迹树网络。这种安全轨迹树网络既可以用于运动规划,也能用于轨迹预测,形成了预测规划多任务框架,从而在整体上提高运动规划方法的性能。为了方便叙述,本章将自动驾驶汽车也简称为“自车”。

% 提出了一种基于学习的框架,该框架利用平滑且运动学可行的轨迹树进行规划。轨迹树减少了交通违规,并可以预测周围的轨迹,作为提高性能的辅助任务。我们的方法采用新的离散可解释输入,然后通过Transformer编码器捕捉它们的空间交互。规划轨迹由轨迹树中得分最高的叶轨迹和Transformer编码器的输出生成。此外,在训练过程中应用了新的焦点损失,以解决自动驾驶汽车长时间保持静止的惯性问题。我们使用CARLA仿真器提供的具有挑战性的场景来评估闭环城市驾驶环境中的方法。实验结果表明,我们的方法在保持高效率的同时优于其他基于学习的方法。自动驾驶的潜在好处,包括减少交通事故、提高交通效率、减少环境污染以及为残疾人提供更大的机动性,使其成为学术界和工业界的热门研究课题。 

\xsection{引言}{Introduction}
自动驾驶汽车技术具有减少交通事故、提高交通效率、降低环境污染并为残疾人提供移动能力的优势。经过近十年的快速发展,该技术已经开始融入到人们的日常生活中\cite{AjanovicLSSH18}。当前自动驾驶汽车在不同的交通场景下执行点对点的导航任务时,整体框架仍然遵循移动机器人系统“感知-规划-控制”的基础设计。在确定全局路线后,运动规划部分负责生成车辆实时运行所需的短期无碰撞轨迹。运动规划常常和行为决策耦合后在整个框架中发挥着关键性的作用。解决运动规划问题需要同时考虑多种约束,主要包括车辆动力学方面的可行性约束,周围动态环境的实时交互约束,红绿灯等交通规则约束以及完成任务所需的时间效率约束等。这些不同约束条件的叠加使运动规划任务变得非常复杂,也使自动驾驶汽车做出合理的决策变得困难。因此,运动规划任务目前仍然存在许多开放性的挑战需要解决。

传统的运动规划方法主要通过引入人工设计的目标函数并采用各种优化方式求其最优解的方式完成驾驶任务。这些传统方法可以分为基于图搜索的方法\cite{lattice}、基于采样的方法\cite{KuwataFTFH08}、基于优化的方法\cite{9716824}和基于曲线的方法\cite{GonzalezPLMN14}。这些方法具有高度的可解释性,并且可以直接发现和纠正自动驾驶汽车运行中的故障。然而,传统运动规划方法的主要问题是严重依赖基于规则的启发式设计。这种设计存在以下两个问题:1)启发式的规则设计可能难以推广到全场景的自动驾驶中;2)大量基于规则的人工设计有可能给规划任务带来不必要的复杂性。

近年来,随着神经网络在各种自动驾驶相关应用中取得了显著成果,出现了许多基于神经网络学习的运动规划方法。基于学习的运动规划方法可以分成两类,包括基于强化学习的方法和基于模仿学习的方法。其中,基于强化学习的方法\cite{9718218,roach,9028159,CaiWSL22}使用奖励函数可以自动获得专家级的驾驶体验,而且无需对复杂的驾驶环境进行显式建模。然而,设计同时满足所有约束的奖励函数非常具有挑战性。在奖励函数设计不合理的情况下,基于强化学习的方法可能难以收敛,从而极大地增加了开发设计具体运动规划算法的难度。

模仿人类专家驾驶策略的模仿学习方法非常适合解决运动规划问题。因为大部分人类驾驶员拥有相当于机器人领域专家级的驾驶技术。大量基于模仿学习的运动规划方法将原始传感器数据直接映射成车辆运动规划轨迹或者车辆可直接执行的控制指令,这些方法也称为端到端方法。如文献\parencite{lav}中所述,端到端方法仍然遵循自动驾驶汽车“分而治之”的策略,即依赖于包含感知、规划和控制的三级模块化框架。早期基于模仿学习的运动规划方法\cite{carla,cilrs,lbc}完全依赖于通过专家演示直接学习低级控制命令。这样会造成过拟合问题,使其泛化能力低下。为了解决该问题,许多基于多任务框架的模仿学习方法\cite{neat,transfuser,lav}应运而生。这些多任务学习框架首先提取出高维栅格特征,然后将高维栅格特征作为多任务框架的共享特征映射到规划主任务和其他多种不同的辅助任务中,包括语义分割、高精度地图信息预测和目标检测等。这些额外的辅助任务增强了共享的高维栅格特征提取辅助任务相关信息的能力,提高了规划主任务的泛化性能。然而,当前的多任务框架主要关注于提高感知模块的性能,忽略了规划模块的基本约束条件,例如舒适性和运动学约束,造成了输出轨迹不可行的问题。作为规划模块输入的高维栅格化特征也存在可解释性低和推理速度慢的问题。此外,基于模仿学习的运动规划方法仍然无法解决“惯性问题”\cite{cilrs},即自动驾驶汽车在复杂场景中保持长期静止不动的现象。

% 本文旨在设计一种高性能、高效率、高安全性和鲁棒性的规划方法。特别地,我们提出了一种
本章重点关注规划模块安全性以及自动驾驶整体任务的完成性,提出了基于模仿学习的多任务运动规划框架,利用安全轨迹树网络(safe Trajectory Tree Network,TTNet)解决上述问题。TTNet网络的关键模块是预定义的轨迹树。轨迹树不仅能够满足车辆运动学约束并显式表达自动驾驶汽车不同的行为意图,而且能够同时作用于核心的运动规划任务以及轨迹预测辅助任务,从而使TTNet网络具备了探索轨迹预测与运动规划之间交互的能力。具体而言,TTNet网络将传统运动规划方法中的可解释性输入表达方式引入基于学习的运动规划方法框架中。可解释性输入表达包括感知结果的边界框以及从全局路线获得的局部参考线。这种表达方式可以很容易地从端到端方法的辅助任务获得,也可以从传统自动驾驶整体框架的感知模块中获得。因此,TTNet网络可以作为端到端方法的一部分或者传统运动规划方法的替代方案,被无缝衔接到自动驾驶整体框架中。TTNet网络引入空间Transformer编码器对可解释性输入的交互特征进行提取。该过程不仅可以融合自动驾驶汽车与其周围环境之间的空间交互信息,而且可以捕捉所有动态智能体与局部参考线之间的交互信息。在完成信息交互后,编码器输出的自动驾驶汽车部分通过注意力机制对轨迹树上的每一条轨迹进行评估,得分最高的叶轨迹将会经过精炼过程生成出最终的规划轨迹。编码器深度特征输出的周围智能体部分的深度特征则通过分类器完成相应智能体的多模态轨迹预测。此外,TTNet网络在训练过程中采用了新的焦点损失函数,以鼓励自动驾驶汽车更高效地完成导航任务,从而有效地缓解了“惯性问题”。

% 。概况及
本章的贡献总结如下:
\begin{enumerate}
    \item 提出了基于模仿学习的多任务规划框架,利用基于Transformer的主干网络来精准捕捉不同交通参与者以及自动驾驶汽车局部参考线之间的交互信息。
    \item 构造了具有曲率连续性和运动学可行性的轨迹树,不仅可以作为规划输出,而且可以用于轨迹预测,从而形成预测规划多任务框架。
    \item 引入了可解释性的输入表达,从而使TTNet网络很容易地接入到自动驾驶整体任务中。同时设计了一种焦点损失函数,有效增强了TTNet网络处理复杂场景的能力。
    \item 相较于最好的方法,TTNet网络在CARLA Town05闭环测试基准上的驾驶分数提高了8.3\%,路线完成度上提高了39.2\%;在CARLA Longest6基准上,TTNet网络的遵守交通规则的分数提高了10.6\%。同时,TTNet网络还具有至少28赫兹的推理速度,比最好的方法快了1.5倍。
\end{enumerate}

\xsection{相关工作}{Related Work}
\paragraph{基于学习的运动规划方法}
目前,大部分基于学习的运动规划方法采用端到端的方式,输入原始的传感器数据,输出运动规划轨迹或者控制命令。这些方法可以分为基于强化学习的和基于模仿学习的方法两种。强化学习是一种解决自动驾驶训练数据集分布变化问题的有效方式。作为首创性工作,Kendall等人\cite{KendallHJMRALBS19}将深度强化学习方法应用到了真实的自动驾驶汽车中。CIRL\cite{cirl}使用深度确定性策略梯度(Deep Deterministic Policy Gradient Algorithms,DDPG)在合理约束的行为空间上探索可能规划路径,克服了大型连续行为空间的低效率探索问题。模仿学习通过监督专家数据集以学习能够完全模拟专家行为的策略。NMP\cite{ZengLSSYCU19}在生成的高维栅格化表征上对预定义的候选轨迹簇进行采样,然后选择出具有最小成本的候选轨迹作为输出。基于模仿学习的方法\cite{p3,mp3,stp3}设计了针对安全性和舒适性的成本函数,可以从专家轨迹簇中选出适当的规划轨迹。尽管许多专注于感知模块性能增强的方法\cite{lav,transfuser,transfuser-cvpr}在CARLA仿真器的闭环测试中取得了良好的性能,但是这些方法仍然存在计算效率低和可解释性弱的问题。

少数基于学习的运动规划方法放弃了端到端的方式,专注于规划模块的性能提高。一部分方法\cite{pip,LiuZUY21,plant,lookout}尝试将周围交通参与者的预测集成到规划中,从而考虑了周围交通参与者与自动驾驶汽车之间的交互影响。这些方法通过设计考虑交互的成本函数,从一组候选轨迹集中评估出最佳的规划轨迹。但是这些成本函数的计算同样存在效率低的问题。部分方法\cite{chauffeurnet,dipp}通过直接预测自动驾驶汽车环境未来时刻热度图的形式,描述了自动驾驶汽车及其周围环境整体的变化,隐式考虑了自动驾驶汽车与环境的交互。

\paragraph{规划的输入表达方式}
鸟瞰图(BEV)能够在避免遮挡和尺度失真等问题的情况下保留3D场景布局信息,非常适合基于学习的规划任务。BEV表达主要从LiDAR、高精度地图和相机图像等获得。部分基于学习的运动规划方法\cite{chauffeurnet,roach}将图像输入隐式投影到BEV中。因为没有BEV真值监督投影过程,使得这些方法获得的BEV质量无法得到保证。Loukkal等人\cite{loukkal2021driving}利用图像和BEV平面之间的单应性将图像显式投影到BEV中。Hu等人\cite{hu2021fiery}执行具有深度估计和图像内参的投影,展现了良好的性能。与之不同的是,传统运动规划方法\cite{emplanner,KuwataFTFH08}使用感知的结果作为输入,主要包括目标语义信息、轨迹预测信息和红绿灯状态信息等。除此之外,传统运动规划方法的输入还包括全局任务路径信息以及定位信息。利用这些信息,运动规划方法能够准确判断当前周围环境中动态交通参与者对于自动驾驶汽车的威胁程度,从而保证了安全性。因此,TTnet网络将传统运动规划方法的输入表达方式引入自身基于学习的规划框架中。

\paragraph{树在规划和预测中的应用}
Lavelle等人\cite{rrt}与1998年首次将快速探索随机树(RRT)引入了机器人规划中。之后,许多RRT变体被广泛用于无人机\cite{9525335}、机械臂\cite{MengGK22}和自动驾驶汽车\cite{9419761}等领域。麻省理工学院的DARPA城市挑战车辆\cite{KuwataFTFH08}对标准的RRT算法做了大量扩展,使RRT能够在具有不稳定动力学的自动驾驶汽车上在线使用。该方法还计算了车辆可行驶空间的成本图,使自车在面对不确定性和有限感知范围时保持安全性。除了规划部分外,基于树的思想也被应用于轨迹预测中。CL-RRT通过RRT将周围交通参与者的意图信息纳入概率性预测轨迹中。Tom等人\cite{jurgenson2019sub}应用起点和终点之间的中间状态作为锚点,然后递归地在每个子路段上生成子锚点,直至得到完整的预测轨迹。SIT\cite{sit}提出了一种用于行人轨迹预测的可解释树。该树使用简单的匀速模型生成,不满足车辆运动学约束条件,因此不能直接应用于自动驾驶汽车的规划中。

\xsection{安全轨迹树网络}{Proposed TTNet Model}
本章的目标是开发一个运动规划算法框架,使自动驾驶汽车能够在复杂的城市交通场景中安全有效地执行点对点导航任务。本章采用深度模仿学习的运动规划方法实现此目标。当前CARLA等大型仿真器能够提供大量专家数据,非常适合基于模仿学习的方法开发。模仿学习是指从专家$\pi^*$中学习得到行为策略$\pi$的过程。本章具体的专家数据集采用观测-运动轨迹对集合的形式。每一个观测-运动轨迹对包含自动驾驶汽车收集到的传感器观测$\mathcal{O}$和其行驶的真实运动轨迹$\mathcal{Y}$两项。未来真实轨迹$\mathcal{Y}$将作为传感器观测$\mathcal{O}$的目标标签,通过监督学习的方式探索从观测到运动轨迹的广义映射。模仿学习的第一步是收集到大量观测-运动轨迹对集合作为专家数据集。
\begin{equation}
    \mathcal{D} = {(\mathcal{O}^i, \mathcal{Y}^i)}_{i=1}^Z
\end{equation}
式中:\textit{Z}表示数据集的大小。

在收集到足够大的专家数据集后,可以通过损失函数$\mathcal{L}$优化求得最终的行为策略$\pi$。
\begin{equation}
    \min_{\pi} \sum_{(\mathcal{O}^i, \mathcal{Y}^i) \in \mathcal{D}} \mathcal{L}(f(\mathcal{O}^i;\pi), \mathcal{Y}^i))
\end{equation}


TTNet网络的整体框架如图\ref{figure:5_arc}所示。TTNet网络的核心是利用轨迹树和神经网络的组合来完成自动驾驶汽车的规划任务以及周围交通参与者的多模态轨迹预测任务。

\begin{figure}[H]
\centering
\includegraphics[width=1\textwidth]{5_arc.pdf}
\caption{TTNet网络示意图}
\label{figure:5_arc}
\centering
\end{figure}

首先,自动驾驶汽车和周围交通参与者的轨迹树集合在离线数据处理阶段被生成出来。该集合能够表示所有交通参与者的离散化可行空间。自动驾驶汽车的轨迹树由多层感知器(MLP)编码得到轨迹树的深度特征编码。自动驾驶汽车及其周围交通参与者的边界框特征和局部参考线也被分别映射成深度特征编码。这些深度特征编码之间经过级联后被送入空间Transformer编码器中。自车、周围交通参与者和局部参考线三种类型信息之间在空间Transformer编码器经过充分的特征融合后,输出整体交互编码。然后,整体交互编码中的自车深度特征编码与红绿灯和目标点信息进行融合。利用注意力机制得到融合后的自车深度特征编码关于轨迹树的得分向量\textit{p}。所获得的得分向量由标签\textit{q}进行监督学习。标签\textit{q}是根据轨迹树中的每条轨迹与自车真实轨迹之间通过轨迹测量操作计算获得的。根据得分向量\textit{p}从轨迹树的深度特征编码中挑选得分最高的轨迹编码与融合后的自车编码执行级联操作,送入GRU后输出最终的运动规划轨迹。除了规划任务外,TTNet网络应用了两个辅助任务以增强网络对于相关信息的处理能力。第一个辅助任务是多模态轨迹预测任务。整体交互编码中周围交通参与者的部分通过由MLP和Softmax层组成的分类器输出周围交通参与者关于其自身轨迹树的得分向量集合$\textbf{\textit{P}}_{pred}$,然后采用与规划任务类似的过程进行监督学习。得分排名靠前的轨迹树分支用作周围交通参与者的概率性多模态预测轨迹。第二个辅助任务是利用MLP将自车融合后深度特征编码输入映射成自车的当前速度。

\xsubsection{输入输出表达}{Input Output representations}
TTNet网络采用自上而下的二维坐标系。原点是自车车顶上的激光雷达传感器的位置。这个坐标系确保自车在当前时刻始终位于原点。同时,自上向下的二维视图提供了表示交通参与者之间空间位置关系简单且易于解释的方式。这种特性对于TTNet网络的泛化性能至关重要。

端到端的方法通常将原始传感器数据转换为深度特征表达作为规划模块的输入。这些深度特征表征方式往往缺乏可解释性,阻碍了规划模块的有效性\cite{chauffeurnet,plant}。TTNet网络选择更简单的离散化表征方式。该表征由5个部分组成的。

1)智能体观测特征集合($\textit{\textbf{O}}_{ob}$):表示当前时刻不同类型交通参与者的位置特征和边界框特征集合。
\begin{equation}
    \textit{\textbf{O}}_{ob} = \{o_{ob}^i\}^N_{i=0}, \ o_{ob}^i = (a^i, box^i, v, \theta^i, w^i, h^i) \in \mathbb{R}^{14}
\end{equation}
式中:$\textit{box} \in \mathbb{R}^{10}$表示自车或感知范围内交通参与者边界框的四个角点位置和中心点位置集合;$o_{ob}^i$表示单个交通参与者\textit{i}的完整观测信息,包括类别\textit{a}、边界框位置点集合\textit{box}、速度\textit{v}、航向角$\theta$、宽度$w$和长度\textit{h};自车的索引号被固定设为$i = 0$,感知范围内交通参与者的索引号设为$i \in \{1, \cdots, N\}$,交通参与者的总数\textit{N}会随着场景和时间变化而变化。$\textit{\textbf{O}}_{ob}$包含了所有智能体的完整信息,对碰撞检测至关重要。

2)交通信号灯状态$\textit{\textbf{O}}_{tl} \in  \mathbb{R}^1$:当前时刻自车传感器检测到的红绿灯信号布尔值。

3)目标点$\textit{\textbf{O}}_{tp} \in \mathbb{R}^2$:全局路径提供中的位于自车前方的短期目标位置。

4)局部参考线:对全局路径上离自车当前位置一定距离范围内的路径点使用样条曲线插值后得到的路径点集合。为了帮助自车沿着局部参考线前进,在每个参考线路径点的左右两侧各手动生成一个角点。角点和参考线路径点之间的距离是自车宽度的一半。整个局部参考线可以表示为:
\begin{equation}
    \textbf{O}_{rf} = \{\textbf{o}_{rf}^i\}^{N'}_{i=0}, \ \textbf{o}_{rf}^i = (a^i, points^i, w^i,) \in \mathbb{R}^{8}
\end{equation}
式中:$o_{rf}$表示单个参考线路径点与其对应左右角点的特征集合,包括参考线的类型\textit{a}(a被强制设为$a=9$)、参考线路径点与左右角点的位置$\textit{points} \in \mathbb{R}^{6}$以及车道宽度$w$;$N'$表示参考线路径点的数量。

5)轨迹树:根据自车和周围交通参与者当前状态信息快速生成的轨迹树。单个轨迹树可以表示成$\textit{\textbf{Y}}_{tree} \in \mathbb{R}^{K \times T \times 2}$,其中\textit{K}是整个树包含的叶轨迹的数量,$\textit{T} = 4$指预测时间长度。

TTNet网络采用的离散表征可以使用标准的检测和跟踪算法从自动驾驶汽车感知任务中获取。许多端到端的方法 \cite{lav,transfuser,interfuser}设计的辅助任务也能够提供这些输入表达。因此,TTNet网络能够无缝集成到端到端框架或标准自动驾驶整体框架中。通过在闭环场景中测试TTNet网络的性能可以确保其在实际自动驾驶汽车中的适用性。文献\parencite{PeiCYJ19,TianPJR18}研究表明,与原始传感器数据相比,如相机图像,使用中间检测结果作为规划输入可以帮助学习到更鲁棒的驾驶行为。

TTNet网络的输出有三部分,包括自车的规划轨迹$\hat{\textit{\textbf{Y}}} = {(x^0_t, y^0_t)}_{t=1}^T$、自车的当前速度$\hat{\textbf{v}}$、以及评价周围交通参与者对应轨迹树的得分向量集合$\textbf{\textit{P}}_{pred}$。自车规划的轨迹会被传递给控制器,生成车辆驾驶所需的转向、油门和制动命令。

\xsubsection{轨迹树的生成}{The Generation of Trajectory Tree}
当前基于学习的运动规划方法缺乏确保规划轨迹曲率连续性和运动学可行性的能力。这种不符合车辆约束的自车轨迹会导致车辆出现各种交通违规行为,尤其是动态复杂环境中常见的碰撞行为。为了解决这个问题,TTNet网络提出了一种新的轨迹树,可以使网络提供可靠性高、鲁棒性强的运动规划轨迹。

轨迹树的生成过程分为两阶段,即扩展阶段和精炼阶段。整个过程是在训练阶段前离线完成的。单个轨迹树的简化示意图如图\ref{figure:5_tree}所示,完整版本的轨迹树如图\ref{figure:5_arc}所示。下面以自车轨迹树的生成过程为例进行说明。

\begin{figure}[H]
\centering
\includegraphics[height=5.8cm]{5_tree.pdf}
\caption{轨迹树生成过程的简化示意图}
\label{figure:5_tree}
\centering
\end{figure}

\paragraph{扩展阶段}
根据自车可能的方向盘转角离散值变化$\{\delta_1, \cdots, \delta_J\}$和速度离散值$\{\hat{v}_1, \cdots, \hat{v}_I\}$生成一组速度-转向对$\{(\hat{v}_{i'}-\delta_{j'})\}_{i'=1,j'=1}^{I,J}$。设定自车的当前位置$(x_0^0, y_0^0)$、速度$v_0^0$和航向角$\theta_0^0$组成的当前状态为根状态。然后,将所有速度-转向对依次带入车辆运动学模型中,得到根节点向指定速度和转角拓展的中间节点。单个中间节点拓展过程可以表示为:
\begin{equation}
\begin{split}
    \theta_1^0 &= \theta_0^0 + \delta_1, \\
    x_1^0 &= x_0^0 + v_0 cos(\theta_1^0) \times dt, \\
    y_1^0 &= y_0^0 + v_0 sin(\theta_1^0) \times dt \\
\end{split}
\end{equation}
式中:$dt$是中间节点和根节点的时间间隔。

为了避免指数爆炸问题,TTNet网络利用了多时间间隔,而不是单步时间进行扩展。在实践中,TTNet网络使用两个时间步长作为间隔。随后,每个中间节点采用相同过程进行拓展,直至得到最终的叶节点。由此得到初步的简化轨迹树。
\paragraph{精炼阶段} 
在拓展过程结束后,TTNet网络引入标准欧拉螺旋线来优化轨迹树结构,提高轨迹树质量。给定根节点和叶节点的状态,轨迹树中的每条轨迹都可以重新表示为标准欧拉螺旋线。
\begin{equation}
\begin{split}
    x &= \Sigma_{n=0}^{+\infty} \frac{(-1)^n a^{2n} l^{4n+1}}{(2n)! (4n+1) 2^{2n}}, \\
    y &= \Sigma_{n=0}^{+\infty} \frac{(-1)^n a^{2n+1} l^{4n+3}}{(2n+1)! (4n+3) 2^{2n+1}}
\end{split}
\end{equation}
式中:$(x, y)$是弧长$l$处的位置;$a$表示曲率的变化率。

使用欧拉螺旋线提炼轨迹树的过程能够确保轨迹的曲率连续性。在得到一组标准欧拉螺旋线组成的轨迹树后,将离散的时间步长映射到欧拉螺旋上就可以获得最终的离散点轨迹树。自车轨迹树通过旋转和平移操作即可获得周围交通参与者的轨迹树。

\xsubsection{空间Transformer编码器}{Spatial Transformer Encoder}
如图\ref{figure:5_arc}展示,空间Transformer编码器不仅用于提取自车和周围交通参与者之间的空间交互特征,还用于提取局部参考线$\textit{\textbf{O}}_{rf}$和智能体观测特征集合$\textit{\textbf{O}}_{ob}$之间的空间交互特征。空间Transformer编码器由8组相同的网络层组成,每层网络可以分成两个网络子层。第一个子层是多头注意力子层,第二个子层是多层前馈网络子层。每个子层之间都顺序经过残差连接\cite{residualconnection}和层标准化\cite{layernormalization}。

\paragraph{输入处理}
首先,局部参考线$\textit{\textbf{O}}_{rf}$和智能体观测特征集合$\textit{\textbf{O}}_{ob}$被两个独立的嵌入层处理。
\begin{equation}
    \textit{\textbf{E}}_{rf} = \mathcal{F}_{rf}(\textit{\textbf{O}}_{rf}), \
    \textit{\textbf{E}}_{ob} = \mathcal{F}_{ob}(\textit{\textbf{O}}_{ob}) \
\end{equation}
式中:$\textit{\textbf{E}}_{rf} \in \mathbb{R}^{N \times d_m}$和$\textit{\textbf{E}}_{ob} \in \mathbb{R}^{{N'} \times d_m}$分别是$\textit{\textbf{O}}_{rf}$和$\textit{\textbf{O}}_{ob}$的深度特征编码;$d_m = 512$是特征维度;$\mathcal{F}_{rf}$和$\mathcal{F}_{{ob}}$是两个全连接层网络。

然后,$\textit{\textbf{E}}_{rf}$和 $\textit{\textbf{E}}_{ob}$之间执行级联操作,得到空间Transformer编码器的输入。
\begin{equation}
    \textit{\textbf{E}}_{ob+rf} = (\textit{\textbf{E}}_{rf} \oplus \textit{\textbf{E}}_{ob}) \in \mathbb{R}^{(N+N') \times d_m}
\end{equation}
式中:$\oplus$表示级联操作。$\textit{\textbf{E}}_{ob+rf}$可以提供自车、周围交通参与者和局部参考线三种不同类型的信息。

\paragraph{多头注意力子层}
多头注意力子层使用自注意力机制处理当前时刻空间信息的输入特征$\textit{\textbf{E}}_{ob+rf}$。与第三章\ref{sptial_att}小节中的空间注意力子层相似,可以将自注意力机制理解成消息传递机制。以$\textit{\textbf{E}}_{ob+rf}$中任意对象\textit{i}的深度特征编码$\textit{\textbf{e}}^i \in \mathbb{R}^{1 \times d_m}$为例,其对应的查询$\{{\textit{\textbf{q}}}^i_h\}_{h=1}^{N_h}$,键$\{{\textit{\textbf{k}}}^i_h\}_{h=1}^{N_h}$和值$\{{\textit{\textbf{v}}}^i_h\}_{h=1}^{N_h}$能够通过不同的全连接层网络独立获得。其中$\textit{\textbf{q}}^i_h \in \mathbb{R}^{1 \times d_q}$,$\textit{\textbf{k}}^i_h \in \mathbb{R}^{1 \times d_k}$和$\textit{\textbf{v}}^i_h \in \mathbb{R}^{1 \times d_v} \ (h \in \{1,\cdots,N_h\})$。$d_q$,$d_k$ 和$d_v$是三个特征维度。$N_h = 8$表示头的数量。第\textit{h}个版本的查询、键和值可以表示为:
\begin{equation}
        \textit{\textbf{q}}^i_h = \mathcal{F}_{q_h}({\textit{{\textbf{e}}}}^i_h),\
        \textit{\textbf{k}}^i_h = \mathcal{F}_{k_h}({\textit{{\textbf{e}}}}^i_h),\
        \textit{\textbf{v}}^i_h = \mathcal{F}_{v_h}({\textit{{\textbf{e}}}}^i_h)\
\end{equation}
式中:$\mathcal{F}_{q_h}(\cdot)$、$\mathcal{F}_{k_h}(\cdot)$和$\mathcal{F}_{v_h}(\cdot)$是三个全连接层网络。

从对象\textit{j}到对象\textit{i}的消息被定义为:
\begin{equation}
    \textit{\textbf{M}}^{j \to i} = \{\textit{\textbf{m}}^{j \to i}_h\}_{h=1}^{N_h}, \ \textit{\textbf{m}}^{j \to i}_h = {\textit{\textbf{q}}^i_h}^T \cdot \textit{\textbf{k}}^j_h.
\end{equation}

此时,对象\textit{i}的注意力头可以表示为:
\begin{equation}
    Att^i_h = Softmax(\frac{\{\textit{\textbf{m}}^{j \to i}_h\}_{j=1}^{N+N'}}{\sqrt{d_k}})\textit{\textbf{v}}_h^i,
\end{equation}
式中:$\sqrt{d_k}$用于提高梯度稳定性。

级联\textit{i}所有的$N_h$个注意力头后通过线性映射得到多头注意力$MultiHead(\textit{Q}^i,\textit{K}^i,\textit{V}^i)$。
\begin{equation}
        MultiHead(\textit{Q}^i,\textit{K}^i,\textit{V}^i) = \mathcal{F}_o(Concat(Att^i_0,\cdots,Att^i_{N_h})). 
\end{equation}
式中:$MultiHead(\textit{Q}^i,\textit{K}^i,\textit{V}^i) \in \mathbb{R}^{1 \times d_m}$,$\mathcal{F}_o(\cdot)$表示全连接层网络;\textit{Concat}表示级联操作;$d_q = d_k = d_v = d_m/N_h$。

此时,对象\textit{i}收集到了包括其自身在内的所有交通参与者和局部参考线所传递的消息。将消息传递过程应于同一场景中的所有对象,可以完成其相互之间的深度特征信息交互。在第二个网络子层中,所有深度特征信息在多层前馈网络子层中学习更高维度的特征。空间Transformer编码器输出更新后的整体交互编码集合。
\begin{equation}
\hat{\textit{\textbf{E}}}_{ob+rf} = \{\hat{\textit{\textbf{e}}}^i\}_{i=0}^{N+N'} \ (\hat{\textit{\textbf{e}}}^i \in \mathbb{R}^{1 \times d_m})
\end{equation}
该集合提供了多任务分支所需的自车和周围交通参与者的深度特征信息。

\xsubsection{多任务框架}{Multi-task Architecture}
\paragraph{规划主任务}
首先,整体交互编码集合的自车深度特征编码$\hat{\textit{\textbf{e}}}^0$通过全连接层网络进行压缩降维。然后,将目标点信息$\textit{\textbf{O}}_{tp}$、红绿灯状态信息$\textit{\textbf{O}}_{tl}$和降维后的自车编码特征级联,产生用于规划主任务的自车深度特征编码$\bar{\textit{\textbf{e}}}^0$。
\begin{equation}
\bar{\textit{\textbf{e}}}^0 = (\hat{\textit{\textbf{e}}}^{0,d_m \to d'_m} \oplus \textit{\textbf{e}}_{tl} \oplus \textit{\textbf{e}}_{tp}) \in \mathcal{R}^{d'_m+3}
\end{equation}
该自车编码能够提供规划主任务所需的基本深度特征。

另一方面,自车轨迹树$\textit{\textbf{Y}}^0_{tree}$通过多层感知器生成包含所有叶轨迹深度特征编码的树编码集合。
\begin{equation}
     \textit{\textbf{E}}^0_{tree}  =  \{\textit{\textbf{e}}^0_k\}_{{k}=1}^K  = MLP(\textit{\textbf{Y}}^0_{tree})
\end{equation}
式中:$\textit{\textbf{e}}^0_k$是第\textit{k}个叶轨迹的深度特征编码。

为了对每条叶轨迹进行有效评估,TTNet网络在自车编码$\bar{\textit{\textbf{e}}}^0$和树编码$\textit{\textbf{E}}^0_{tree}$之间应用注意力机制,生成自车编码关于树编码的得分向量\textbf{\textit{p}}。
\begin{equation}
    \textbf{\textit{p}} = Softmax(\{\mathcal{F}_{ego}(\bar{\textit{\textbf{e}}}^0)\cdot \mathcal{F}_{tree}(\textit{\textbf{e}}^0_k)^T\}_{{k}=1}^K)
\end{equation}
式中:$\mathcal{F}_{ego}$和$\mathcal{F}_{tree}$是两个全连接层网络;\textit{T}表示矩阵转置。

在轨迹树中,最接近未来真实轨迹的叶轨迹提供了自车行为的最合理解释,需要为其赋予最高的得分。因此,TTNet网络采用轨迹测量操作,计算轨迹树中所有叶轨迹和未来轨迹真值之间的$L_2$距离。计算完成后,最接近未来真实轨迹的叶轨迹索引号被设置为\textit{\textbf{q}},并用其对得分向量\textbf{\textit{p}}进行监督学习。

如文献\parencite{cilrs}所述,模仿学习方法受到“惯性问题”的影响严重。惯性问题是指自车在训练数据中长期保持静止,从而使自车学习到次优的静止策略问题。为了解决这个问题,TTNet网络将焦点损失函数引入训练过程中。
\begin{equation}
    \mathcal{L}_{score} = -\sum^{K}_{k=1}(\alpha_k (1-\textit{\textbf{p}}_k)^{\gamma} log(\textit{\textbf{p}}_k))
\end{equation}
式中:$k$表示叶轨迹的索引;$\alpha_k$用于调整不同标签的损失权重;$\gamma = 2$用于平衡标签的分布。

在训练中,真实静止轨迹的$\alpha$被设置为0.25($\alpha_{k=1} = 0.25$),其他索引值的真实轨迹$\alpha$被设置为0.75($\alpha_{k!=1} = 0.75$)。这样的设定使得静止轨迹数据变成了“背景数据”,其他轨迹变成了“前景数据”,从而有效地减少了静态真实轨迹的损失贡献,缓解了训练数据中存在大量静止轨迹而导致的惯性问题。

在训练阶段,自车编码$\bar{\textit{\textbf{e}}}^0$和真实标签的叶轨迹编码$\textit{\textbf{e}}^0_q$级联,然后使用GRU自回归式生成最终的规划轨迹。
\begin{equation}
    \hat{\textit{\textbf{Y}}} = GRU((\bar{\textit{\textbf{e}}}^0 \oplus \textit{\textbf{e}}_q))
\end{equation}
在推理阶段,真实标签的叶轨迹编码$\textit{\textbf{e}}^0_q$被替换成拥有最高得分的叶轨迹编码$\textit{\textbf{e}}^0_{top1}$,然后采用相同的过程生成规划轨迹。

\paragraph{轨迹预测辅助任务}
除了完成主要的规划任务外,TTNet网络还引入了预测周围交通参与者未来多模态轨迹的辅助任务。拥有预测周围交通参与者轨迹的能力对自车做出合理决策至关重要。轨迹预测辅助任务可以为空间Transformer编码器主干网络提供自车做出合理决策所需的自车与周围交通参与者之间的动态交互信息。多模态轨迹预测辅助任务的输入是整体交互编码集合中周围交通参与者的编码部分,称为预测编码${\textit{\textbf{E}}}_{pred} = \{\hat{\textit{\textbf{e}}}^i\}_{i=1}^{N}$。与规划任务的复杂设计不同,轨迹预测任务仅使用包含多层感知器和Softmax层的分类器输出周围交通参与者对应轨迹树的得分向量。
\begin{equation}
    \textbf{\textit{P}}_{pred} = Softmax(MLP(\hat{\textit{\textbf{E}}}_{pred}))
\end{equation}

距离周围交通参与者未来真实轨迹最近的叶轨迹索引集合$\textbf{\textit{Q}}_{pred}$也通过轨迹测量操作获得的。多模态轨迹预测任务在训练时同样使用了焦点损失函数。
\begin{equation}
    \mathcal{L}_{pred} = \mathcal{L}_{focal}(\textbf{\textit{P}}_{pred}, \textbf{\textit{Q}}_{pred})
\end{equation}
其中,$\mathcal{L}_{focal}$表示轨迹预测的焦点函数。

\paragraph{整体损失函数}
TTNet网络的整体损失函数包含四个部分:规划焦点损失函数$\mathcal{L}_{score}$、预测焦点损失函数$\mathcal{L}_{pred}$、规划回归损失函数$\mathcal{L}_{plan}$和当前速度回归损失函数$\mathcal{L}_{spd}$。

在规划任务中,回归损失函数可以表示为:
\begin{equation}
    \mathcal{L}_{plan} = \mathcal{L}_{reg}(\hat{\textit{\textbf{Y}}},{\textit{\textbf{Y}}})
\end{equation}
式中:$\mathcal{L}_{reg}$表示Smooth$L_1$损失函数\cite{fastrcnn}。
% \begin{equation}
% \begin{split}
%     \mathcal{L}_{reg} &= \sum_{t=1}^{T}(\mathcal{L}_t^x+\mathcal{L}_t^y)/2T,//
%     \mathcal{L}_t^x &= 
% \end{split}
% \end{equation}

为了提供自车对于当前状态的估计能力,TTNet网络引入了当前自车速度预测的辅助任务。该任务通过将自车编码输入到多层感知器中来预测自车的当前速度$\bar{\textit{\textbf{e}}}^0$
\begin{equation}
    \hat{v}^0 = MLP(\bar{\textit{\textbf{e}}}^0)
\end{equation}
自车当前速度的预测也使用Smooth$L_1$损失函数。
\begin{equation}
    \mathcal{L}_{spd} = \mathcal{L}_{reg}(\hat{v}^0, v^0).
\end{equation}

TTNet网络利用多损失函数的加权组合训练多任务模型。整体损失函数表示如下:
\begin{equation}
    \mathcal{L} = \lambda_{score} \mathcal{L}_{score} + \lambda_{pred} \mathcal{L}_{pred} + \lambda_{plan} \mathcal{L}_{plan} + \lambda_{spd} \mathcal{L}_{spd}
\end{equation}
式中:$\lambda_{score}$、$\lambda_{pred}$、$\lambda_{plan}$和$\lambda_{spd}$为各损失项的权重。

\xsection{实验结果与分析}{Experimental Results and Analysis}
本节分别从闭环测试相关设置、与基准方法的性能比较、可视化结果和消融实验等方面对TTNet网络的性能进行全面分析评估。

\xsubsection{数据集、评价基准和指标}{Datasets, Evaluation Benchmarks and Metrics}
\paragraph{数据集}
开源的CARLA仿真器能够帮助开发性能强大的自动驾驶算法,非常适合作为测试环境评估TTNet网络在复杂交通场景下的规划能力。与之前的版本相比,CARLA仿真器0.9.10.1版本性能更加稳定,因此选用该版本进行实际的训练数据集收集和闭环测试。CARLA仿真器0.9.10.1版本具有10个公开可用的城镇,能够帮助基于模仿学习的方法收集足够多的专家数据。TTNet网络使用CARLA仿真器本身提供的基于规则的自动驾驶运动规划方法建立110K的训练数据集。

\paragraph{评价基准}
TTNet网络使用两个闭环测试基准进行性能评估,分别是Town05基准和Longest6基准。

Town05基准是Town01-10中最特别的环境基准。因为该基准具有最广泛的驾驶场景,包含了单车道、多车道、合流车道等多种车道类型以及高速公路、城市道路、桥梁和地下通道等多种驾驶场景。参照文献\parencite{transfuser-cvpr},TTNet网络使用两种评估设置。1)短任务路线(Town05 Short Routes):包含32条短路线,每条路线长度均为0.07公里,并且整条路线至少经过了3个十字路口。2)长任务路线(Town05 Long Routes):包含10条长路线,每条路线长度约为1-2公里,并且整条路线至少经过了10个十字路口。这两种设置中的所有路线中都包含有高度密集的动态交通参与者。在任务路径沿线预定义的位置还会生成具有对抗性的场景,例如在交通路口和环岛的交互通行等。因为Town05长任务路线比Town05短任务路线更为复杂,因此更加具有挑战性。

Longest6基准是文献\parencite{transfuser}于2022年提出的。该基准更加贴近真实交通流,而且比之前的测试基准挑战性更强。Longest6基准包括36条路线,每条路线的长度为1.5公里,这些路线均匀分布在Town01-06中。每条路线都独立设置其气候和光照条件,并在任务路径沿线上设置了大量行人、摩托车和车辆的动态交互。与Town05基准相似,Longest6在任务路径沿线的预定义设置了包括紧急让行等各种对抗场景。

\paragraph{评价指标}
为了与其他方法进行公平比较,TTNet网络设定了五项的评价指标,通过评价多种驾驶行为以体现不同运动规划方法的性能。路线完成度、违规得分和驾驶得分是三项主要的评价指标。此外,TTNet网络还提供每公里的违规行为和得分率两项评价指标。每公里的违规行为能够为自动驾驶汽车提供更全面的违规分析。得分率是自动驾驶汽车在每种特定类型的对抗场景下更直观的性能展现。所有评价指标是针对每条任路径单独计算的,然后统计出测试基准所有路径的平均值作为最终的性能对比指标。

路径完成度(Route Completion,RC):实际行驶长度与路线总长度的比率。
\begin{equation}
    R^{RC}_i = \frac{L^{actual}_i}{L^{total}_i}
\end{equation}
式中:$L^{actual}_i$和$L^{actual}_i$分别表示自动驾驶汽车实际的行驶长度和路线总长度,$R_i$表示单个任务路径\textit{i}的路径完成度。

违规得分(Infraction Score,IS):违规处罚系数的几何级数。
\begin{equation}
    P^{IS}_{i} = \prod_j^{Red,\cdots,Veh}(p^j)^{\# \ infractions_j}
\end{equation}
式中:$p^j$是自动驾驶任务中发生违规事件\textit{j}的惩罚系数。基础分数的初始值设定为\textit{1.0}。每发生一次违规行为,则基础分数衰减一次,衰减系数为该违规行为的惩罚系数。每个违规行为的处罚系数都是预定义的,与行人碰撞时的处罚系数设置为0.50,与车辆碰撞时的惩罚系数设置为0.60,与静态布局碰撞时的罚系数设置为0.65,违反红灯时的罚值设置为0.7。

驾驶得分(Driving Score,DS):违规处罚和路线完成度之间的乘积。
\begin{equation}
    DS_i = P^{IS}_{i} \cdot R^{RC}_i.
\end{equation}
驾驶得分是反映不同运动规划方法驾驶性能的综合性指标。


每千米违章数(Infractions per km):每项违规的总次数除以总里程数。
\begin{equation}
    Infractions \ per \ km = \frac{\sum_i^I \# \ infractions_i}{\sum_i^I k_i},
\end{equation}
式中:$k_i$是路径\textit{i}的驾驶距离(\textit{km});\textit{I}是任务路径总量。交通违规类型包括与行人发生碰撞(Ped)、与车辆发生碰撞(Veh)、与静态元素(Stat)发生碰撞、闯红灯违规(red)、驶出道路违规(off)、驶离任务路线违规(RD)、路线超时(RT)和车辆堵塞(Block)。

每千米得分率(Scoring Rate per km):根据每种交通违规类型的每公里违规次数计算得到的总惩罚系数。
\begin{equation}
    SR = (p^j)^{Infractions \ per \ km},
\end{equation}
式中:$j$表示任意一种交通违规类型。

\xsubsection{基准方法介绍}{Baselines}
TTNet网络在两种评价基准上与大量基准方法进行了比较,主要包括:
\begin{itemize}
    \item CILRS\cite{cilrs}:以高阶导航命令为条件,直接将单目摄像头采集到图像转换成车辆可以直接执行的控制命令。
    \item LBC\cite{lbc}:将模仿学习的过程分解为两个阶段,第一阶段以感知数据的真值训练一个特权智能体,并在第二个阶段将其作为老师来训练一个常规智能体。常规智能体可以从输入图像中直接提取出车辆控制命令。
    \item NEAT\cite{neat}:通过使用中间注意力图迭代地将高维2D图像特征压缩为紧凑表示,并开发了一种连续函数将BEV场景坐标中的位置映射到路径点和语义目标中。
    \item Roach\cite{roach}:使用强化学习训练出比基于规则的专家性能更好的专家策略,并提供特有的监督信号供模仿学习运动规划方法训练出更加有效的驾驶策略。
    \item WOR\cite{wor}:假设自动驾驶汽车及其动作不影响周围环境的其他交通参与者,然后设计代价函数计算所有候选动作值的得分,得分最高的动作值作为实际的规划轨迹发送给车辆执行。
    \item ST-P3\cite{stp3}:提出了一种时空特征学习方法,以提取感知、预测和规划任务所需的更具表达力的鸟瞰图特征。
    \item TransFuser\cite{transfuser-cvpr}:提出了一种新的传感器融合方法,从而利用注意力机制能够有效地集成图像和激光雷达的特征。
    \item TransFuser*\cite{transfuser}:是TransFuser的增强版本,采用了新的传感器配置,通过多相机设置增加了自动驾驶汽车的视觉感知范围。
    \item TCP\cite{tcp}:将轨迹规划输出和控制指令输出相结合,实现两种输出方式的优势互补。
    \item LAV\cite{lav}:是当前性能最好的基于学习的运动规划方法。该方法不仅学习了自动驾驶汽车的驾驶策略,还学习了周围车辆的驾驶行为。这种学习方式能够在无需收集额外数据的情况下通过学习更多的样本而产生性能更好的专家策略。
\end{itemize}

\xsubsection{实施细节}{Implementation Details} 
为了能够与其他端到端方法在闭环测试中进行对比,TTNet网络在运动规划方法的基础上设计了感知和控制模块,使得TTNet网络能够完成整体自动驾驶任务。

\paragraph{感知模块}
感知模块采用基于TransFuser*\cite{transfuser}的主干网络。该主干网络能够利用激光雷达和相机融合的方法提供TTNet网络输入。其中,激光雷达只使用传感器前半部分的数据,包括距离车前32m和两侧16m范围内的点。这些点会转换为二维BEV网格的形式与相机提供的图像数据进行融合。相机安装在距离自车原点前方1.5m、上方2m的地方,分辨率为900$\times$256,视野为100°。针对TTNet网络的输入,感知模块在主干网络上使用CenterNet\cite{centernet}解码器的关键点估计来定位周围交通参与者,使用门控单元解码器来预测局部参考线,同时使用多层感知器解码器来检测当前的交通灯状态。在数据收集过程中,相机位置可以平移和旋转,以实现数据增强。平移区间为[-1,1],旋转区间为[-5°,5°]。在训练预处理步骤中相应地调整了周围交通参与者和激光雷达点的位置和方向。感知模块是独立训练的,在与其他模块联合测试时感知模型参数保持不变。

\paragraph{控制模块}
在规划模块生成自车的未来轨迹后,跟踪控制器可以将轨迹转换为车辆能够直接执行的驾驶命令。TTNet网络使用了两个比例积分微分(proportional-integral-derivative,PID)控制器分别进行纵向和横向控制。纵向PID控制器的作用是将当前速度调整为目标速度,并在红灯或接近周围交通参与者时刹车。目标速度$v_{target}$是根据下一时刻$t=1$和下下一时刻$t=2$位置之间的$L_2$距离得到的。
\begin{equation}
    v_{target} = ||(x_{2},y_2) - (x_1,y_1)||^2
\end{equation}
当目标速度低于$0.1km/h$时,纵向PID控制器开始进行制动刹车控制。另一方面,横向PID控制器通过获取当前转向角与目标转向角的差值来不断调整驾驶方向,目标转向角被设置为位于下一时刻$t = 1$和下下一时刻$t = 2$中间位置的航向。在闭环测试实验中,纵向PID控制器的参数设置为$K_P=5.0$、$K_I=0.5$和$KD=1.0$。横向PID控制器的参数设置为$K_P = 1.25$、$K_I = 0.5$和$KD = 0.3$。

\xsubsection{定量结果与分析}{Quantitative Results and Analysis}

\paragraph{Town05基准性能对比实验}
Town05基准的实验结果展示在表\ref{town5benchmark}中。其中平均值和标准差是基于三次重复实验计算得到的。ST-P3的结果是从原始论文中获得,该论文只提供了平均值。此外,每种方法规划模块的输入也显示在表中。

从表\ref{town5benchmark}中可以看出,TTNet网络使用完全不同的规划输入,取得了最高的路径完成度(RC)和驾驶得分(DS)。这样的结果表明TTNet网络具有更强的自动驾驶性能。在Town05短任务路线上,TTNet网络比当前性能最好的方法LAV在驾驶得分上高出8.3\%。在Town5长任务路径上,TTNet网络比LAV在路径完成度上显著提高了39.2\%。与具有类似感知模块的TransFuser相比,TTNet网络在Town05短任务路线上驾驶得分提高了67.9\%,路径完成度上提高了21.6\%。在Town05长任务路线上,TTNet网络比TransFuser驾驶得分提高了65.5\%,路径完成度提高了69.8\%。

\begin{table}[H]
\centering
\caption{TTNet网络与基准方法在Town05基准上的性能对比结果}
\begin{tabularx}{\textwidth}{*{7}Y}
\toprule
\multirow{2}*{方法}  &\multirow{2}*{出处}  & \multirow{2}*{规划输入} & \multicolumn{2}{c}{Town05 Short Routes} &\multicolumn{2}{c}{Town05 Long Routes} \\ \cline{4-7}
& & &DS $\uparrow$            &RC $\uparrow$            &DS $\uparrow$              &RC $\uparrow$       \\ \midrule
CILRS\cite{cilrs}    &ICCV  &BEV  & 7.47 ± 2.51    & 13.40 ± 1.09      & 3.68 ± 2.16     & 7.19 ± 2.95    \\
LBC\cite{lbc}   &CoRL &BEV   & 30.97 ± 4.17   & 55.01 ± 5.14      & 7.05 ± 2.13     & 32.09 ± 7.40   \\ 
NEAT\cite{neat}  &ICCV &BEV    & 58.70 ± 4.11   & 77.32 ± 4.91      & 37.72 ± 3.55    & 62.13 ± 4.66   \\
Roach\cite{roach} &ICCV  &BEV    & 65.26 ± 3.63   & 88.24 ± 5.16      & 43.64 ± 3.95    & 80.37 ± 5.68   \\
WOR\cite{wor}  &ICCV   &BEV     & 64.79 ± 5.53   & 87.47 ± 4.68      & 44.80 ± 3.69    & 82.41 ± 5.01   \\
TransFuser\cite{transfuser-cvpr} &CVPR &BEV & 54.52 ± 4.29   & 78.41 ± 3.75      & 33.15 ± 4.04    & 56.36 ± 7.14   \\ 
ST-P3\cite{stp3} &ECCV  &BEV     & 55.14   & 86.74      & 11.45    & 83.15   \\ 
LAV\cite{lav} &CVPR  &BEV   & 84.37 ± 1.01   & 92.67 ± 1.03      & 54.24 ± 3.19    & 68.73 ± 3.80   \\ 
\midrule[0.5pt]
TTNet &  & $\textbf{\textit{Y}}_{Tree}$+$\textbf{\textit{O}}_{ob+rf}$ &\textbf{91.56} ± 0.81  &\textbf{95.31} ± 0.31 &\textbf{54.87} ± 2.78 &\textbf{95.68}  ± 1.96\\ 
\bottomrule
% Expert   &  &  & & 84.67 ± 6.21   & 98.59 ± 2.17      & 38.60 ± 4.00    & 77.47 ± 1.86   \\ \hline
\end{tabularx}
\label{town5benchmark}
\end{table}

图\ref{fig:sr}展示了TTNet网络和基准方法在Town05短任务路线中的每千米得分率。如图所示,TTNet网络在红绿灯、静态障碍物避障和车辆避障方面获得了100\%的得分率。具体来说,TTNet网络在红绿灯方面的性能比LBC提高了98.2\%。在静态障碍物避障方面性能比LAV提高了33.6\%。在车辆避障方面的性能比LBC提高了86.2\%。另一方面,TTNet网络在避让行人方面的得分率为51.1\%,低于其他基准方法。

\begin{figure}[H]
\begin{center}
\includegraphics[height=5.4cm]{5_sr.pdf}
\caption{Town05短任务路线每千米得分率}
\label{fig:sr}
\end{center}
\end{figure}

\paragraph{Longest6基准性能对比实验}
表\ref{longest6benchmark1}展示了TTNet网络与基准方法在Longest6基准上的主要评价性能对比结果。与Town05基准相似,所有实验均在相同的设置下重复三次。与基准方法相比,TTNet网络在Longest6基准上取得了最高的违章得分(IS)和驾驶得分(DS)。这样的结果再次表明了TTNet网络自动驾驶综合性能更强。从具体数据对比来看,TTNet网络在DS、RC和IS上的性能比最好的方法LAV分别提高了5.6\%、2.8\%和10.6\%。与具有类似感知模块的Transfuse相比,TTNet网络在DS和IS上分别提高了26.5\%和46.0\%。

\begin{table}[H]
\centering
\caption{TTNet网络与基准方法在Longest6基准上的主要评价指标对比结果}
\begin{tabularx}{\textwidth}{*{4}Y}
\toprule
\multirow{2}*{方法}  & \multicolumn{3}{c}{性能}  \\ \cline{2-4}
                     &DS $\uparrow$       &RC $\uparrow$     &IS $\uparrow$   \\ \midrule[0.5pt]
WOR\cite{wor}  & 20.53 ± 3.12 & 48.47 ± 3.86 & 0.56 ± 0.03 \\
TCP\cite{tcp}  &44.26 ± 4.39 &69.96 ± 4.93 &0.61 ± 0.04  \\ 
TF*\cite{transfuser} & 47.30 ± 5.72 &\textbf{93.38} ± 1.20 &0.50 ± 0.06 \\ 
LAV\cite{lav}  &56.65 ± 1.16 &82.61 ± 0.99 &0.66 ± 0.02 \\  \midrule[0.5pt]
TTNet &\textbf{59.82} ± 2.14 &84.88 ± 2.21 &\textbf{0.73} ± 0.05 \\ \midrule[0.5pt]
Expert &76.91 ± 2.23 &88.67 ± 0.56 &0.86 ± 0.03 \\ 
\bottomrule
\end{tabularx}
\label{longest6benchmark1}
\end{table}

表\ref{longest6benchmark2}中展示了在Longest6基准上TTNet网络与基准方法的每千米违章数对比结果。与其他基准方法相比,TTNet网络在Longest6基准上每千米违章数都处于最低水平。这样的结果表明TTNet网络具有更高的安全性。从具体数据上讲,在与行人碰撞和闯红灯方面,TTNet网络的表现明显优于LAV,分别高出92.5\%和96.1\%。在与惯性问题直接相关的任务超时和阻塞上,TTNet网络的性能比LAV分别提高了77.8\%和38.5\%。TTNet网络在这两项上的性能甚至比专家分别提高了48.0\%和38.5\%。

\begin{table}[H]
\centering
\caption{TTNet网络与其他方法在Longest6基准上每千米违章数对比结果}
\begin{tabularx}{\textwidth}{*{9}Y}
\toprule
\multirow{2}*{方法}  & \multicolumn{8}{c}{性能 }  \\ \cline{2-9}
        &Ped $\downarrow$  &Veh $\downarrow$    &Stat $\downarrow$ &Red $\downarrow$   &Off $\downarrow$ &RD $\downarrow$ &RT $\downarrow$ &Block $\downarrow$ \\ \midrule[0.5pt]
WOR\cite{wor}  & 0.18 & 1.05 & 0.37 & 1.28 & 0.47 & 0.88 & 0.08 & 0.20  \\
TCP\cite{tcp}  &\textbf{0.00} &0.10 &0.72 &0.22 &0.19 &0.00 &0.08 &0.45  \\ 
TF*\cite{transfuser} &0.03 &2.45 &0.07 &0.16 &0.04 &0.00 &0.06 &0.10   \\ 
LAV\cite{lav}  &0.04 &0.08 &0.20 &0.33 &\textbf{0.00} &0.07 &0.09 &0.13  \\ \midrule[0.5pt]
TTNet &0.003 &\textbf{0.07} &0.10 &\textbf{0.013} &0.047 &\textbf{0.00} &\textbf{0.02} &\textbf{0.08}   \\  \midrule[0.5pt]
Expert &0.02 &0.28 &0.01 &0.03 &0.00 &0.00 &0.08 &0.13   \\
\bottomrule
\end{tabularx}
\label{longest6benchmark2}
\end{table}

TTNet网络在Town05和Longest6基准测试上取得高性能的主要原因有三个:1)与端到端方法规划模块中使用的高维BEV输入相比,TTNet网络的离散化输入方式,即轨迹树、交通参与者边界框表达和局部参考线,具有更强的场景表达能力。2)TTNet网络引入了轨迹树进行规划和轨迹预测,使所有具有非完整性约束的交通参与者都能获得曲率连续性和运动学可行性的轨迹,不仅增加了自动驾驶汽车的驾驶性能,而且更加精准地描述了交通参与者的运动特征。3)基于Transformer的主干网络以及针对规划的多任务设计,通过强大的消息交互机制显著减少了自动驾驶汽车的碰撞,增强了自动驾驶汽车做出安全决策的能力。

\paragraph{轨迹树的表达能力}
为了说明TTNet网络构建的轨迹树本身对于自动驾驶汽车的专家轨迹和周围交通参与者的未来真实轨迹具有较强的表达能力,表\ref{ade_fde}展示了轨迹树在TTNet网络和部分基准方法收集到的专家数据集上的结果。其中规划轨迹指自动驾驶汽车未来的真实轨迹,预测轨迹指周围交通参与者未来的真实轨迹。利用轨迹预测中常用的平均位移误差(ADE)和最终位移误差(FDE)来衡量轨迹树的性能。

\begin{table}[H]
\centering
\caption{关于轨迹树表达能力的性能实验}
\begin{tabularx}{\textwidth}{*{7}Y}
\toprule
 &\multirow{2}*{性能/米} & \multicolumn{5}{c}{数据集}   \\ \cline{3-7} 
                        &            &LBC  &LAV &Transfuser &TCP  &TTNet\\ \midrule
规划 &$ADE$        &0.08  &0.10  &0.10      &0.15      &0.14           \\ \cline{2-7}
轨迹                        &$FDE$     &0.11  &0.18  &0.16      &0.19      &0.21   \\ \midrule[0.5pt]
预测&$ADE$         &0.57  &0.60  &0.54      &0.71     &0.66           \\ \cline{2-7}
轨迹                 &$FDE$         &0.70  &0.80  &0.68     &0.92    &0.89   \\ \midrule[0.5pt]
\multicolumn{2}{c}{数据集大小}     &157K  &400K  &228K      &420K      &1.1M           \\ \bottomrule
\end{tabularx}
\label{ade_fde}
\end{table}

从表\ref{ade_fde}的结果可以看出,TTNet网络离线生成的轨迹树与所列专家数据集的规划轨迹之间的ADE和FDE都非常小。这表示轨迹树具有良好的迁移性和关于自车轨迹的丰富表达能力,可以适用于所列的所有专家数据集。尽管轨迹树与专家数据集中的预测轨迹之间ADE和FDE较规划轨迹较高,但是整体预测误差仍然限制在较小的误差区间内。这说明轨迹树对于周围交通参与者的预测性能虽然低于对于自车的规划能力,但是仍然能够满足轨迹预测的需求。

\paragraph{计算效率实验}
运动规划方法需要满足自动驾驶计算负担和计算时间的要求。表\ref{inferencetime}展示了TTNet网络与基准方法的推理时间对比。所有闭环仿真实验均是在同一台包含GeForce RTX 2080Ti GPU的设备上测试的。因为PID控制器的时间被限制在0.6ms以内,因此表\ref{inferencetime}未显示TTNet网络下游控制模块的时间。实验结果表明,TTNet网络在Town05基准上的推理速度是58赫兹,在Longest6基准上的推理速度是28赫兹。两者推理速度相差比较大的原因是因为Town05基准每个场景中设置的动态交通参与者数量少于Longest6基准。TTNet网络搭建的自动驾驶整体算法框架的推理速度可以在Town05基准上达到15赫兹,在Longest6基准上达到11赫兹。与LAV和Transfusser*相比,TTNet网络的推理速度分别快了1.5$\times$和3$\times$。TTNet网络的高效率使其能够完全满足自动驾驶的实时性要求。

\begin{table}[H]
\centering
\caption{TTNet网络与基准方法推理时间的对比}
\begin{tabularx}{\textwidth}{*{4}Y}
\toprule
 \multirow{2}*{方法} & \multicolumn{3}{c}{推理时间/秒}  \\     \cline{2-4}  
            &Town05 Short  &Town05 Long  &Longest6\\ 
\midrule[0.5pt]
Transfuser         & 0.184        & 0.183       & 0.183 \\ 
LAV                & 0.104        & 0.098       & 0.097 \\ \midrule[0.5pt]
Perception         & 0.046        & 0.046       & 0.050 \\  
TTNet(Planning)    & 0.018        & 0.017       & 0.035 \\
Total              &\textbf{0.064}&\textbf{0.063}&\textbf{0.085} \\
\bottomrule
\end{tabularx}
\label{inferencetime}
\end{table}

\xsubsection{可视化结果与分析}{Qualitative Results and Analysis}
定量分析结果表明,TTNet网络在DS、RC和IS方面优于其他基准方法。本节提供TTNet网络不同场景的可视化结果展示。在所有场景中,中心位置显示了自动驾驶汽车的规划轨迹,右下角显示了BEV视角下的自车及周围交通参与者的情况。

图\ref{fig:case}展示了自动驾驶汽车接近红绿灯路口的例子。开始时,自动驾驶汽车沿车道正常行驶,TTNet网络输出正常巡航状态的规划轨迹。当交通灯从绿色变为红色时,TTNet网络能够检测到红绿灯状态的变化,并输出完全静止的自车轨迹。之后场景展示中的图例与图\ref{fig:case}的图例一致。

\begin{figure}[H]
\begin{center}
\includegraphics[width=0.98\textwidth]{5_tra.pdf}
\caption{红绿灯场景}
\label{fig:case}
\end{center}
\end{figure}

图\ref{subfig:2}中展示了TTNet网络在跟车场景下的结果。在尚未进入拥堵路段时,自动驾驶汽车以正常速度行驶。当自动驾驶汽车行驶至拥堵路段时,TTNet网络根据离散化周围交通参与者输入的变化,有效地减缓了自动驾驶汽车的速度,并使其无缝地汇入到整体车流中。
\begin{figure}[H]
\begin{subfigure}[b]{0.49\linewidth}
\centering
\includegraphics[height=5.8cm]{5_3.pdf}
\subcaption{正常行驶状态}
\end{subfigure}
\begin{subfigure}[b]{0.49\linewidth}
\centering
\includegraphics[height=5.8cm]{5_4.pdf}
\subcaption{跟随状态}
\end{subfigure}
\caption{车辆跟随场景}
\label{subfig:2}
\end{figure}

图\ref{subfig:3}展示了自动驾驶汽车在无保护左转路口场景的结果。首先,自动驾驶汽车沿着既定的路线通过一个无保护的左转路口。此时,左转方向有直行车辆正在通过路口。按照交通规则,自动驾驶汽车需要停车等待。TTNet网络同样输出了停车等待的运动规划轨迹。然后,当检测到直行车辆整体通过局部参考线后,自动驾驶汽车恢复行驶速度,并继续沿指定路线行驶。这种场景也突出体现了局部参考线在轨迹规划中的重要性。

\begin{figure}[H]
\begin{subfigure}[b]{0.49\linewidth}
\centering
\includegraphics[height=5.8cm]{5_5.pdf}
\subcaption{左转等待状态}
\end{subfigure}
\begin{subfigure}[b]{0.49\linewidth}
\centering
\includegraphics[height=5.8cm]{5_6.pdf}
\subcaption{左转通行状态}
\end{subfigure}
\caption{左转场景}
\label{subfig:3}
\end{figure}

图\ref{subfig:4}展示了自车在多车道道路上行驶时在固定地点执行换道操作的场景。

综上所述,这些场景示例说明了TTNet网络在处理日常生活中常见的驾驶场景中具备高安全性和高鲁棒性的优点。

\begin{figure}[H]
\begin{subfigure}[b]{0.49\linewidth}
\centering
\includegraphics[height=5.8cm]{5_7.pdf}
\subcaption{正常行驶状态}
\end{subfigure}
\begin{subfigure}[b]{0.49\linewidth}
\centering
\includegraphics[height=5.8cm]{5_8.pdf}
\subcaption{换道状态}
\end{subfigure}
\caption{换道场景}
\label{subfig:4}
\end{figure}

\xsubsection{消融实验}{Ablation Study}
本节在Longest6基准进行了一系列的消融实验研究,评估了TTNet网络中各个组分的有效性。消融实验结果如表\ref{5_ablation}所示。需要评估的TTNet网络具体组分包括针对规划的轨迹树、焦点损失函数、多模态轨迹预测辅助任务、针对预测的轨迹树和当前速度预测辅助任务的作用,这些组分在表\ref{5_ablation}中的简化表示为“PlanT”、“Focal”、“PredL”、“PredT”、“Speed”。表中显示的结果是三次重复实验的平均值。

\begin{table}[H]
\centering
\caption{消融实验结果}
\begin{tabularx}{\textwidth}{*{9}Y}
\toprule
 \multirow{2}*{ID}&\multicolumn{5}{c}{成分} & \multicolumn{3}{c}{性能} \\ \cline{2-9}
 & PlanT &Focal &PredL & PredT  &Speed  &DS &RC &IS \\ \midrule
(1) &$\times$ &$\times$ &$\times$  &$\times$  &$\times$   &41.07  &61.77 &0.67 \\
(2) &simple  &$\times$ &$\times$ &$\times$  &$\times$   &42.36  &65.08 &0.67 \\
(3) &\checkmark &$\times$ & $\times$ &$\times$  &$\times$    &46.10  &66.59 &0.70 \\
(4) &\checkmark &plan & $\times$ &$\times$  &$\times$   &48.26  &70.36 &0.69 \\
(5) &\checkmark & plan &\checkmark &$\times$ &$\times$  &52.73  &75.56 &0.69 \\
(6) &\checkmark & plan &\checkmark &\checkmark  &$\times$  &57.18  &78.63 &0.70 \\
(7) &\checkmark & \checkmark &\checkmark &\checkmark  &$\times$  &58.44  &80.88 &0.71 \\ \midrule[0.5pt]
(8) &\checkmark & \checkmark &\checkmark &\checkmark &\checkmark &59.82 &84.88 &0.73 \\
\bottomrule
\end{tabularx}
\label{5_ablation}
\end{table}

总体而言,TTNet网络所有变体的性能都低于完整框架ID8。ID1仅使用3层的多层感知器将融合后的自车深度特征编码映射成运动规划轨迹,导致DS从59.82急剧下降到41.07。ID2使用简单的全连接层网络和Softmax层组成的分类器替换多层感知器,以分类的方式选择轨迹树中具有最高得分的候选轨迹作为运动规划轨迹。与ID1相比,ID2的性能略有改善。完整的针对规划的轨迹树被加入到ID3的模型中,性能与之前的变体相比有了显著的改进。ID4将焦点损失应用于规划轨迹树。可以观察到ID4的性能有小幅改善。与规划部分的消融实验类似,ID5将轨迹预测辅助任务加入到模型中,ID6将完整的用于规划和预测的轨迹树加入到模型中,ID7将用于预测的交叉熵损失替换为焦点损失,其性能较之前的变体都有所增加。这些消融研究结果表明TTNet网络的每个组件在实现高性能方面都发挥着重要作用。


\xsection{本章小结}{Chapter Summary}

本章提出了一种基于模仿学习的运动规划方法,称为安全轨迹树网络(TTNet网络)。针对之前基于学习的运动规划方法无法提供可解释性强的规划模块输入问题,TTNet网络将传统运动规划方法的离散表达方式引入自身框架中。针对之前基于学习的运动规划方法无法保证运动规划轨迹的运动学可行性的问题,TTNet网络提出了轨迹树的概念,不仅可以有效地表达包括自动驾驶汽车在内的不同类型交通参与者的不同行为意图,并且能够提供具有运动学可行性和曲率连续性的运动规划结果。同时,TTNet网络采用空间Transformer编码器成功提取了自动驾驶汽车、周围交通参与者和局部参考线的交互信息,其输出的整体交互编码能够同时用于规划任务和辅助的多模态轨迹预测和当前速度预测任务中。在训练过程使用的焦点损失函数也进一步增强了TTNet网络的安全性。综上所述,TTNet网络利用离散表达方式、基于Transformer的主干网络和轨迹树在CARLA仿真器上高性能、高效率、高安全性和高鲁棒性地完成了自动驾驶规划任务。