% !TeX root = ../main.tex

\xchapter{绪论}{Introductions}

\xsection{研究背景与意义}{Research Background and Significance}

城市交通是指行人、非机动车、车辆等交通方式在人口稠密的城市环境下沿不同地点之间的流动。高效便捷的城市交通能够支撑就业、教育、医疗保健等行业的高速运转,对于现代社会的发展至关重要。车辆作为现代城市中的核心交通工具,得到了广泛的应用。车辆的大量应用也引起了三个问题。第一个是交通拥堵问题。在上下班、节假日、恶劣天气等特殊时段会出现大量车辆挤满整条道路的情况,造成交通缓慢甚至堵塞,导致交通运行效率低下。第二个是环境污染问题。尽管当前电动汽车正在慢慢普及,但是燃油车的保有量仍然很大,其排放的一氧化碳、氮氧化物和颗粒物等污染物会对人类健康造成严重影响,同时也是全球气候变暖的主要成因之一。第三个是安全问题,据世界卫生组织2022年统计数据显示,每年约有130万人在交通事故中丧生,平均每24秒就有人会死于交通事故,还有2000万到5000万左右的人受到非致命伤害而引发残疾。自动驾驶汽车技术就是解决这些问题的核心方案之一。

自动驾驶汽车技术起源于20世纪60年代。美国斯坦福大学机械工程学院研究生Adams建造了“Stanford Cart”的自动驾驶小车,用来支持他在视频信息控制远程车辆的问题研究。Stanford Cart可以沿着事先设定好的路线行驶,并避开路上遇见的各种障碍物。这一研究成果是自动驾驶汽车技术的开端。在此后的几十年内,自动驾驶相关技术得到不断地发展改进。在上世纪90年代,梅赛德斯-奔驰公司推出了“Pre-Safe”自动刹车系统。该系统通过一系列传感器检测车辆自身的不稳定性以及潜在的碰撞风险。如果系统认定当前车辆存在碰撞风险,则会发出报警声提醒驾驶员注意,并在必要时和自适应巡航控制系统协同工作,以避免碰撞或降低碰撞的严重程度。进入21世纪后,自动驾驶汽车技术迎来突破性发展。标志性事件是美国国防部高级研究计划局(Defense Advanced Research Projects Agency,DARPA) 在2003年发起的旨在推动自动驾驶汽车技术快速发展的一系列挑战赛,包括2004年和2005年的两届“DARPA Grand Challenge”,以及2007年的“DARPA Urban Challenge”。尤其是2007年的DARPA城市挑战赛,参赛车辆需要完成超车、汇车、泊车以及十字路口停车等一系列任务。这三次DARPA挑战赛极大地推动了自动驾驶技术的发展。在城市挑战赛结束的两年后,谷歌公司宣布了自动驾驶汽车项目研发计划。该项目于2010年开发出搭载激光雷达和相机等传感器的自动驾驶原型车,并在美国加州山景城开放道路上进行了一系列测试。谷歌的自动驾驶汽车公司Waymo在2016年成立,开始提供开放的自动驾驶汽车出行服务。时至今日,自动驾驶汽车技术已经成为了热门研究领域。各大研究机构、汽车制造商和科技公司都在积极探索自动驾驶汽车相关技术的商业化应用。随着自动驾驶技术的普及,拥有高性能、高效率、高安全性和高鲁棒性的自动驾驶汽车能够极大地缓解交通拥堵,改善环境污染,减少交通事故,为城市的智能化发展做出重大贡献。

顾名思义,自动驾驶汽车(也称为无人驾驶汽车)是指在没有任何人为干预的情况下能够自主驾驶的无人移动系统。为此,自动驾驶汽车需要具备感知周围环境,预测环境动态变化趋势,并进行自主决策规划的能力。为了满足这样的需求,自动驾驶汽车上安装有一系列传感器,主要包括激光雷达、相机、毫米波雷达、惯性导航单元等。如图\ref{fig:1_architecture}所示,整个自动驾驶系统可以分为感知、定位、规划和控制四大模块。其中,感知部分负责为自动驾驶汽车提供周围的动静态环境信息,主要包括目标检测、交通标识检测、可行驶区域检测、目标跟踪和轨迹预测等模块。定位部分负责实时提供自动驾驶汽车所在全局坐标系的位置,同时负责构建驾驶任务的高精度地图,主要包括GNSS定位、高精度地图构建和局部地图匹配等模块。规划部分负责提供从起始位置经过多个任务点并最终到达终点的任务路径,并根据自动驾驶汽车当前的环境做出合理的决策,判别交叉路口或换道决策时的行为优先级,然后生成运动轨迹发送给控制系统,主要包括任务规划、行为决策和运动规划。控制系统根据运动轨迹和当前自动驾驶汽车状态生成车辆可直接执行的控制指令,包括横向控制和纵向控制。

\begin{figure}[H]
\centering
% \includegraphics[height=9.8cm]{1_architecture.pdf}
\includegraphics[width=0.8\textwidth]{result-21.png}
\caption{自动驾驶汽车整体框架}
\label{fig:1_architecture}
\end{figure}

经过近些年来的快速发展,自动驾驶汽车技术在感知、定位和控制模块方面进展显著。2D目标检测的检准率已经达到96\%,定位和控制精度也都能够达到十厘米量级。然而,自动驾驶汽车技术在规划方面仍然存在很多问题和挑战。美国加州交通管理局最新公布的2022年自动驾驶脱离报告显示,苹果自动驾驶汽车的年度人工接管次数为5982次,平均每21英里就需要一次人工介入。这些人工介入的直接原因就是规划模块在复杂场景中因为安全性得不到保证而导致的功能失效。

保障规划模块安全性的关键因素之一在于轨迹预测能够提供周围交通参与者未来精准可靠的运动轨迹输入。在最理想的情况下,轨迹预测为自动驾驶汽车提供周围交通参与者未来一段时间内精准的确定性(单模态)行为轨迹即可满足要求。规划模块仅通过单模态的轨迹预测结果就可以准确掌握自身与与其他交通参与者的动态交互情况,从而消除潜在的威胁。然而交通参与者行为本身就具有不确定性,例如面对前方低速行驶的车辆时,激进的驾驶员会选择超车行为,而保守的驾驶员会选择跟车行为。因此,规划模块需要考虑周围交通参与者未来一段时间内多种可能的(多模态)行为轨迹。综合考虑自动驾驶汽车的计算性能和安全性要求,实际的规划模块对于两种不同类型的轨迹预测输入都有需求。对于跟车、汇入车流等简单交通场景,准确可靠的单模态轨迹预测输入即可满足规划要求。对于超车、十字路口、环岛等复杂交通场景,概率性的多模态轨迹预测输入才能保证规划的安全性。然而,当前的单模态轨迹预测方法大部分只适用于同种类别的交通参与者,如行人或者车辆,缺乏预测不同类别交通参与者轨迹的能力。对于更难处理也是最易受伤害的行人,当前的多模态轨迹预测方法仍然不能提供概率性的结果。同时,多模态轨迹预测与规划模块之间的交互仍处于早期探索阶段,尚需开展深入研究。

本文针对自动驾驶轨迹预测和运动规划方法开展研究工作,研究目标在于设计能够满足自动驾驶安全性需求的轨迹预测和运动规划方法,最终实现安全可靠的自动驾驶。针对上述面临的设计挑战,本文采取以Transformer网络为核心框架的设计方法来提高轨迹预测和运动规划的性能,同时兼顾速度。在算法研究中,本文注重从精度、推理速度、可迁移性、自动驾驶整体性能等多方面分析展示算法效果。针对单模态轨迹预测,本文提出了可以预测多种类别交通参与者的时空Transformer网络。针对多模态轨迹预测,提出了高效快速的概率性候选轨迹网络。在此基础上,本文提出了针对运动规划的多任务框架,通过安全轨迹树网络探索轨迹预测与运动规划之间的交互。本文提出的方法对自动驾驶领域具有重要的研究与应用价值。

\xsection{国内外研究现状}{Domestic and International Research Status}

\xsubsection{自动驾驶技术总体研究进展}{Overall Research Progress of Autonomous Driving Technology}
自上世纪60年代开始至今,全球各大互联网公司、汽车生产商和学术研究机构都在积极探索自动驾驶技术相关商业化应用,布局智能汽车相关业务。国外著名的自动驾驶公司有Waymo、优步、特斯拉、Mobileye、英伟达等。国内自动驾驶技术相关的公司包括百度、小马智行、智加科技、华为等。另外,几乎所有的汽车生产商,例如梅赛德斯-奔驰、丰田、通用等,都将自动驾驶作为未来重点的研发方向。国外有名的研究机构包括卡耐基梅隆大学、斯坦福大学、弗吉尼亚理工大学、麻省理工学院等,国内从事自动驾驶研究的高校有西安交通大学、清华大学、国防科技大学、军事交通学院、同济大学等。

自动驾驶感知模块通过处理原始传感器数据,包括视觉图像、激光点云、毫米波数据等,提供周围环境的动静态信息。自从深度学习在计算机视觉领域取得突破性进展,目标检测作为感知模块的核心已趋于成熟。常用的目标检测网络包括Faster R-CNN\cite{RenHGS15},YOLO\cite{RedmonDGF16}和CenterNet\cite{centernet}等。通过合理的传感器配置,目标检测能够覆盖自动驾驶汽车周边360°的范围,目标的检出率和检准率能够达到96\%以上。

自动驾驶定位模块通过融合全球导航卫星系统(Global Navigation Satellite System,GNSS)和惯性导航单元的坐标数据,为自动驾驶汽车提供准确的位置。在空旷无遮挡的环境下,良好的GNSS信号能够使定位误差降至10厘米以下。即使是在卫星信号受限的环境内,应用同时定位与建图算法(Simultaneous Localization And Mapping,SLAM)也能够通过鲁棒性高的匹配算法将定位误差降至15厘米以下,从而保证自动驾驶汽车的正常运行。

车辆控制模块在自动驾驶尚未兴起时已经得到了长足发展。PID控制器(Proportion Integration Differentiation,PID)在自动控制的各个领域得到了广泛应用,并且能够使自动驾驶汽车的横向误差和纵向误差分别降至20厘米和30厘米。随着最近几年模型预测控制(Model Predictive Control,MPC)算法的兴起,控制精度也有了进一步的提升。

% 交通参与者的未来状态可以用未来轨迹表示,用于提前检测潜在危险并用于设计决策或规划算法,如图1所示。但是,由于交通参与者的机动方式不同, 交通参与者与环境的复杂交互、感知信息的不确定性、自动驾驶汽车的计算负担和计算时间要求,如何准确预测交通参与者的未来轨迹备受关注,成为提高安全性的关键点之一
% 最后,当前运动规划与上游轨迹预测的关系仍然
% 最后,目前大部分的规划算法仍然无法很好地处理概率性的多模态轨迹预测输入。

自动驾驶规划模块能够制定任务路线并实时规划出短期运动轨迹,对自动驾驶的安全性和任务完成度至关重要。当前规划模块仍然有很大的局限性。首先,当前大部分实际自动驾驶汽车上的运动规划方法不能提供量化指标,导致各种运动规划方法无法进行性能对比。其次,运动规划实验的可重复性差。即使在相同的输入下重复多次实验,运动规划方法的输出差异也非常大。最重要的是作为规划模块输入之一的轨迹预测方法性能较差,不能满足规划要求。轨迹预测作为自动驾驶感知模块与规划模块的接口,能够提供自动驾驶汽车周围交通参与者未来的行为动态,为下游的规划模块判断当前环境中的潜在危险提供帮助。轨迹预测是建立在感知其他前端模块基础上的。在感知技术趋于成熟的今天,轨迹预测也得到了蓬勃发展。许多自动驾驶公司推出了自己的公开数据集,如nuScenes\cite{nuscenes}、Argoverse\cite{Chang_2019_CVPR}、Aplolloscape\cite{ma2019trafficpredict}等。然而,与控制和定位技术的厘米级误差形成鲜明对比的是,轨迹预测的精度仍然非常低下。例如,Argoverse目前排名第一的QCNet-AV1算法的单模态轨迹预测精度只有1.53米(2023年3月13日)。这意味自动驾驶汽车的安全性受到了轨迹预测方法的严重制约。

综上所述,轨迹预测与运动规划是保证自动驾驶安全性,推动自动驾驶进一步发展的关键技术。

\xsubsection{自动驾驶轨迹预测的研究现状}{Research of Trajectory Prediction for Autonomous Vehicles}
自动驾驶轨迹预测按照建模方式可以分为三大类,包括基于物理的方法、基于模式的方法和基于规划的方法,典型方法如图\ref{fig:1_pred}所示。基于物理的方法定义了基于牛顿运动定律的显式动力学方程。基于模式的方法从观测轨迹数据中学习运动模式。基于规划的方法主要是对交通参与者的运动意图进行推理。

\begin{figure}[H]
\centering
\includegraphics[width=0.98\textwidth]{1_pred.pdf}
\caption{自动驾驶轨迹预测典型方法}
\label{fig:1_pred}
\end{figure}

\subsubsection{基于物理的轨迹预测方法} 
基于物理的方法按照时间顺序通过前向推演动力学方程\textit{f}来预测交通参与者的多步运动。这些动力学方程遵循牛顿运动定律。基本的动力学方程\textit{f}可以表示成$\Dot{s}_t = f(s_t,\textit{\textbf{u}}_t,t) + w_t$。其中,$\textit{\textbf{u}}_t$是未知的控制输入,$w_t$是过程噪声。事实上,基于物理的轨迹预测方法可以被视为从各种估计或观测中推断出$s_t$和$\textit{\textbf{u}}_t$。基于物理的轨迹预测方法又分成单模型轨迹预测方法和多模型轨迹预测方法。

% 具有高斯噪声的匀速度模型如图\ref{fig:1_cv}所示。
单物理模型轨迹预测方法将目标的运动状态表示为位置、速度和加速度,然后使用物理模型\textit{f}进行轨迹预测。主要的单物理模型包括运动学模型、动力学模型和自回归模型。其中,运动学模型具有简单性和稳定性的优点,非常适用于运动不确定性小的短期轨迹预测。运动学模型可以分为匀速模型、具有高斯噪声的匀速度模型、具有高斯噪音加加速度的分段匀加速度模型,具有高斯噪声转弯加速度和线速度的协同转向模型以及通常用作车辆动力学模型近似值的自行车模型\cite{4632283}。Batz等人\cite{5164400}使用变体的协同转向模型对无迹卡尔曼滤波内的车辆进行单步运动预测,并通过与预测车辆之间的距离来判断自动驾驶汽车所处的危险程度。Møgelmose等人\cite{7225707}提出了一种基于纯视觉的行人轨迹预测框架,将有关行人危险区域的信息与所有行人的概率性匀速模型预测位置相结合,向驾驶员警告处于危险中的行人。动力学模型主要考虑在交通参与者上的作用力。当考虑车轮、变速箱或发动机的物理特性时,动力学模型会变得非常复杂。Zernetsch等人\cite{7535484}使用考虑非机动车辆的动力和各种阻力的动力学模型进行轨迹预测。与标准匀速模型相比,该模型2.5秒内的长期预测更加准确。Elnagar等人\cite{725351}使用最大似然估计的三阶自回归模型参数来预测移动物体的下一个位置和方向。Cai等人\cite{cai2006robust}在粒子滤波中引入了二阶自回归模型以预测曲棍球运动员的运动。

然而,交通参与者的复杂运动很难用单物理模型\textit{f}来描述。多模型轨迹预测方法设定了多种原型运动模式,每个模式可以用不同单物理模型\textit{f}或多物理模型的融合进行描述,例如匀速运动或突然加速等。这种设定形成了能够描述复杂运动的行为序列。多模型方法\cite{1561886}通过离散值模态\textit{s}增加连续值\textit{x}的形式维护混合系统状态$\xi = (x, s)$。交互式多模型滤波器(Interactive Multiple Model filter,IMM)是一种广泛应用于多模型方法的推理技术。Kuhnt等人\cite{7535554}考虑IMM的预定义环境几何来估计车辆的可能路线的方法。Xie等人\cite{8186191}将基于运动学的匀转弯速率和加速度模型与基于IMM的车道保持和换道行为相结合。该方法根据道路几何形状产生可变预测范围的结果。动态贝叶斯网络作为混合估计方法的核心囊括了大量来自概率图模型的精确推理技术和建模方案。Gindele等人\cite{5625262}使用动态贝叶斯网络对所有车辆的未来轨迹进行联合建模,并采用数字特征描述多个驾驶员之间的局部交互。这些特征用于对每个驾驶员的当前状况进行分类,并推理出其可能采取的行为。Blaiotta等人\cite{8760356}提出了一种用于行人预测的动态贝叶斯网络,设计了步行和站立两种运动模式。

基于物理的方法利用物理模型以相对较低的计算资源完成轨迹预测。尽管这些方法的精度相对较低,但越来越多的模型使用基于物理的思想来提高精度。由于交通参与者的物理模型不断变化,这些方法中的大多数仅适用于短期预测(不超过1秒)。使用一个或多个物理模型可以快速获得交通参与者的未来轨迹,但物理模型的选择和它们之间的切换将带来明显的预测误差。

% 基于模型的复杂性,我们认识到以下子类
\subsubsection{基于模式的轨迹预测方法}
与基于物理的方法使用明确定义的运动学和动力学函数不同,基于模式的方法从数据中学习参数化函数,并遵循“感知-学习-预测”模式。这类方法使用不同的函数拟合来学习交通参与者的行为模式,这些函数包括高斯过程、隐马尔可夫模型、深度学习网络等。本文主要关注采用深度学习网络的轨迹预测方法,按照深度学习网络类型的不同,分别对序列网络、图卷积神经网络和生成模型三种类型的网络进行介绍。

\paragraph{序列网络模型}
序列网络用于提取历史轨迹的深度特征,主要包括循环神经网络(Recurrent Neural Network,RNN)、卷积神经网络(Convolutional Neural Network,CNN)和注意机制(Attention Mechanism,AM)。

RNN能够存储历史时间内的特征信息,利用输入和隐藏状态确定输出的预测信息。原始的RNN存在梯度爆炸的问题。作为RNN变体,长短期记忆网络(Long Short-Term Memory Network,LSTM)和门控循环单元(Gated Recurrent Unit,GRU)能够很好地解决该问题并有效处理时间序列数据,因此被广泛应用于交通流预测、动作识别、视频处理和轨迹预测等领域。Alahi等人\cite{slstm}提出的Social LSTM是LSTM应用于轨迹预测的里程碑工作,该方法通过池化机制学习行人的社会交互并预测其未来轨迹。Zyner等人\cite{8290702}将三层LSTM作为序列分类器的一部分观察周围车辆的意图,使得自动驾驶汽车能够安全穿过无信号交叉口。Ding等人\cite{8793568}提出了用于城市环境的自动驾驶车辆轨迹预测的在线框架,使用LSTM预测车辆的高阶驾驶策略,例如,前进、让行、左转和右转等。然后该策略依据交互相关因素\cite{yangwenyan}、交通规则和地图信息来优化上下文推理过程。Kawasaki等人\cite{9196738}针对城市环境中交通行为的高度不确定性问题,提出了基于车道的多模态预测方法,通过引入车辆运动学约束和矢量图车道特征来表示车辆状态和车道之间的广义几何关系。该方法能够处理任意形状和数量的车道,并预测出每条车道上的车辆未来轨迹以及每条车道被选择的概率。Deo等人\cite{8500493}针对高速公路上的车辆轨迹预测问题,将六种不同的LSTM编组为六种特定行为,并根据其置信度输出概率最高的行为轨迹。

CNN同样能够处理时间序列信息。Nikhil等人\cite{nikhil2018convolutional}提出了基于CNN的行人轨迹预测方法,在保证性能的同时比LSTM网络的推理速度更快。该方法采用叠加卷积层的方式实现对时间连续性的处理,通过全连接层方式输出未来轨迹。与LSTM只能处理连续的历史轨迹输入不同的是,CNN还能够处理鸟瞰图输入。Caltech等人\cite{covernet}根据车辆速度、加速度和偏航率,生成一组候选轨迹集合。然后通过分析栅格化的输入特征,选择候选轨迹集中概率最高的轨迹作为未来轨迹。Cui等人\cite{9197560}使用CNN与自行车模型结合的方式,解决了模型直接从数据中学习约束会导致输出轨迹运动学上不可行或次优的问题。Jan等人\cite{9341327}对以交通参与者为中心的图像执行卷积操作,为其预测多个可能的未来轨迹。每条轨迹包括位置、速度、加速度、方向、偏航率和不确定性估计。该方法依赖于从先前的对象跟踪阶段估计的不确定性,并采用时间卷积网络提取历史轨迹特征。Gilles等人\cite{9812253}利用高精度地图的图形化表示方式和稀疏投影生成表达未来轨迹的热度图。该热度图是交通参与者的无约束栅格化表征形式,可以有效表示预测的多模态性和预测不确定性。

模仿人类思维方式的注意力机制使用有限的注意力资源从大量信息中快速过滤出高价值的信息\cite{hanhao,liuchuang,sunyasheng,zhangzhiyuan}。Kim等人\cite{9304741}在交互建模中利用无监督的方式关注最有影响力的车辆,并利用多头注意力的编码解码结构预测高速公路上的车辆多模态轨迹。Messaoud等人\cite{9084255}受人类推理启发,使用注意机制学习相邻车辆对未来状态影响的重要程度。该方法模拟了比双车辆交互更高阶的交互,利用对周围车辆的全局和部分注意力组合输出多模态轨迹。Vaswani等人\cite{attention}于2017年提出的Transformer网络是自然语言处理领域的里程碑式工作。该模型使用大量的多头注意力机制来完成机器翻译任务。其独特的注意机制以并行计算的方式处理序列数据,摒弃了RNN串行处理结构。由于Transformer网络在机器翻译中取得了优异的结果,部分研究工作也开始将Transformer网络应用于轨迹预测任务中。Giuliari等人\cite{transformer}首先将Transformer模型和BERT模型用于行人轨迹预测中。实验结果表明,基于Transformer的模型比LSTM尤其是在长期预测中拥有更好性能,并且还可以处理输入数据帧缺失的问题。除了对轨迹序列进行建模,Transformer网络还可以对交通参与者和环境之间的交互进行建模\cite{huang2022multi,li2020end,xiefeng}。Liu等人\cite{mmtransformer}使用叠加的Transformer网络作为主干网络,对环境信息、历史轨迹和高精度地图信息进行了有效提取并输出多模态预测轨迹。

\paragraph{图神经网络模型} 
尽管RNN或CNN的方法在提取欧几里得空间数据特征时取得了成功,但许多实际应用场景中的数据都是从非欧几里得空间生成的。在轨迹预测中,每个交通场景都拥有数量可变的无序节点,可以被视为不规则的图结构。图中的每个节点都有与其他节点相连的边,用以捕捉节点之间的相互依赖性。图神经网络(Graph Neural Network,GNN)能够很好地处理这种无序图结构,非常适合捕捉轨迹预测任务中的时空交互特征。

% 不仅考虑了相同时间内不同交通参与者节点之间的空间交互,还考虑了相同交通参与者在不同时间内的时间交互。
图卷积网络(Graph Convolutional Network,GCN)和图注意力网络(Graph Attention Network,GAT)是两种效果比较好的轨迹预测GNN。因为图结构中每个节点的相邻节点数量不同,导致卷积操作不能直接在图结构上使用。通过学习从节点及其邻居中提取交互特征的映射函数,GCN成功地将卷积运算从图像数据处理扩展到图结构的处理。Li等人\cite{grip}提出了基于时空GCN的轨迹预测算法GRIP,将每个交通参与者每个时刻的特征都视为节点,并考虑了节点之间的相互作用影响。如果两节点为同一交通参与者且时间相邻,则两节点之间存在时间边。如果两节点为不同交通参与者且两者之间的距离小于固定阈值,则两节点之间存在空间边。构建时空图完成后,GRIP通过GCN对时空图特征进行提取,并使用LSTM预测出未来轨迹。GRIP++\cite{grip++}是GRIP的改进版本。GRIP++使用动态图网络来预测交通参与者的轨迹,并采用GRU进行轨迹预测。Chandra等人\cite{ChandraGPMBBM20}将谱图分析和深度学习相结合,使用双层GNN-LSTM结构输出低级的未来轨迹信息和高级的行为信息。与大部分利用鸟瞰图表示轨迹和地图信息的方式不同,Gao等人\cite{lanegcn}采用矢量的方式表示这些信息,通过分层GNN的方式进行轨迹预测,并引入了随机掩码恢复任务提高性能。

\paragraph{生成模型} 
生成模型主要用于多模态轨迹预测,包括生成对抗网络(Generative Adversarial Network,GAN)和条件变分自动编码器(Conditional Variational Auto Encoder,CVAE)两种。

通过生成器和鉴别器的不断博弈进化,生成对抗网络(GAN)可以获得高质量的生成器和判断能力强的鉴别器\cite{wenhuiying,gan,gan_ou}。当GAN应用于轨迹预测时,生成器用于生成预测轨迹,鉴别器用于判断预测轨迹是否正确。Gupta等人\cite{sgan}提出的SGAN网络首次将GAN引入行人轨迹预测中。SGAN的生成器使用LSTM编码器、池化模块和LSTM解码器来生成预测轨迹,鉴别器使用LSTM来确定预测轨迹是否合理。Yang等人\cite{9447207}在SGAN的基础上设计了隐变量预测器,重点关注如何有效提取行人的社会交互关系。Li等人\cite{8967822}使用环境注意机制进行深度特征提取,并利用GRU进行轨迹预测。Sadeghian等人\cite{sophie}同时利用所有行人和车辆的历史信息以及场景上下文信息,将社会注意力机制与物理注意力相结合,帮助模型提取图像中与路径最相关的部分。该方法利用GAN生成真实样本,并通过对其分布建模来捕捉未来轨迹的不确定性。Zhao等人\cite{zhao2019multi}提出了行人张量融合网络,能够对交通参与者的历史运动、社会交互、场景上下文的约束以及行人行为的随机性进行有效建模。

% 因此,Kingma等人[154]提出了
自动编码器以尽量减少重建误差为目标,通过编码器将数据压缩成低维向量表示,并使用解码器对低维向量进行解码获得重构输出。变分自动编码器使用神经网络对变分推理中的分布进行参数化,弥补了自动编码器生成数据能力差的问题。在此基础上,条件变分自动编码器(CAVE)\cite{cave}能够完成包含真实类别信息的结构化预测任务。Lee等人\cite{lee2017desire}首次在轨迹预测任务中引入了CVAE,并通过反向优化控制结构根据预测结果的长期收益选择最优轨迹。Katyal等人\cite{9197434}使用贝叶斯方法预测的目标意图作为CVAE的条件生成多条路径。同时,该方法还引入了LSTM鉴别器以对抗方式训练整体框架。Cheng等人\cite{ChengLYSR20}将场景上下文信息以及个体和群体之间的交互作为CVAE的条件输出多模态轨迹。

% 目前,CNN和LSTM仍然是模拟驾驶行为时空模式的主要模型,而GAN/VAE利用生成学习来捕捉全球的交互行为。近年来,图形神经网络在有效地建模交互方面变得引人注目。
\subsubsection{基于规划的轨迹预测方法}
基于规划的方法将轨迹预测转化为序列决策问题,得到能够推理未来的交通参与者运动模型。与前两种方法不同,基于规划的方法进行运动建模时引入了理性交通参与者的概念。理性假设要求模型考虑当前行为对未来的影响。因此,基于规划的方法不仅考虑单个动作成本,而且使用目标函数考虑一系列动作的总代价。本节主要介绍应用较广泛的逆强化学习(Inverse Reinforcement Learning,IRL)和生成对抗模仿学习(Generative Adversarial Imitation Learning,GAIL)方法。

逆强化学习(IRL)根据专家演示学习奖励函数,以生成相应的最优驾驶策略。IRL根据学习奖励函数权重的方式分为基于最大边际的方法和基于最大熵的方法。基于最大边际的方法通过最小化专家演示和预测轨迹之间的特征期望来优化奖励函数权重。Ratliff\cite{ratliff2006maximum}用结构化的最大化边际学习从特征到奖励的映射,并在马尔科夫链中使用这些最优策略来模仿专家的行为。大多数基于边际的方法存在特征期望匹配上的模糊性问题。与之相比,基于最大熵的方法能够使用多个奖励函数来解释专家行为的模糊性。与先前方法直接学习从单一观测到多个未来轨迹的一对多映射模式不同,Deo等人\cite{deo2020trajectory}允许策略推断合理的目标位置以及二维栅格上通向这些目标的路径。同时,基于注意力的轨迹生成器可以从最大熵逆强化学习的策略中采样状态序列,用以生成未来连续轨迹。

生成对抗模仿学习(GAIL)利用GAN的方式在强化学习中进行模仿学习。GAIL不是从专家演示中学习奖励函数,而是直接从数据中提取策略。和GAN一样,GAIL的核心思想是生成器尽可能生成与专家轨迹相似的轨迹,判别器尽可能判断是否是专家轨迹。Kuefler等人\cite{7995721}将GAIL扩展到循环策略的训练中,并由鉴别器评估策略和行动。Bhattacharyya等人\cite{9990591}将人类驾驶建模为特征是非线性和随机性的序列决策问题,使用参数共享的GAIL扩展版本进行多智能体驾驶行为建模。同时引入奖励增强模仿学习,修改奖励信号以向智能体提供特定领域的知识,从而将GAIL与领域知识结合起来。为了克服GAIL仅使用当前状态对下一个状态进行建模的缺点,Choi等人\cite{choi2021trajgail}在GAIL框架内结合部分可观察马尔可夫决策过程,使用来自鉴别器的奖励函数对模型进行训练。


\xsubsection{自动驾驶运动规划的研究现状}{Research of Motion Planning for Autonomous Vehicles}
运动规划方法可以分成传统运动规划、端到端和交互式运动规划三种,如图\ref{fig:1_plan}所示。传统运动规划方法根据决策结果生成短期的运动轨迹。端到端方法直接将原始的传感器数据通过深度学习网络映射出自动驾驶车辆可行驶的运动轨迹。交互式运动规划方法根据与周围环境交互的结果输出轨迹。

\begin{figure}[H]
\centering
\includegraphics[width=0.85\textwidth]{1_plan.pdf}
\caption{自动驾驶运动规划典型方法}
\label{fig:1_plan}
\end{figure}

% 基于采样的规划器缺点是解决方案不是最优的。
\subsubsection{传统运动规划方法} 
传统运动规划方法主要是指非神经网络运动规划方法。从自动驾驶汽车技术开始至今,传统运动规划方法一直是大部分实际自动驾驶平台的核心方法,也是应用最广泛的运动规划方法,主要分为四种类型。第一种是基于图搜索的规划方法。基于图搜索的方法通过将离散化状态空间内的顶点与可行边连接来构造离散图,然后将规划问题转化成离散图上搜索最优序列的问题。这类方法主要包括Dijkstra算法、A*算法、状态格子算法等。Ajanovic等人\cite{AjanovicLSSH18}设计了在城市环境内能够实时生成上百米区域的规划轨迹。该方法首先建立了能够表达搜索空间和驾驶约束的几何表示方法,然后在成本代价地图上使用A*算法进行搜索以计算最佳运动轨迹。McNaughton等人\cite{5980223}提出了lattice搜索空间表示,允许搜索算法实时探索时空维度,完成跟车、换道和汇入车流等动作。第二种是基于采样的规划方法。这类规划器试图解决高维空间内算法的时间限制问题,包括对配置空间或状态空间进行随机采样并寻找其中的连通性。最常用的采样方法有概率路图法和快速探索随机树(Rapidly-exploring Random Tree,RRT)。前者主要应用在移动机器人中,后者已针对自动驾驶汽车进行了广泛测试。Vasile等人\cite{7989177}综合考虑了路网中运动规划和路线选择中的违反最小约束问题。该方法使用线性时序逻辑公式定义车辆的期望行为,使用基于RRT*的运动规划器来获得场景下的最小违规轨迹。第三种是基于优化的规划方法。这类方法迭代优化解空间,直到满足终止或收敛条件。根据优化的类型,基于优化的方法可以分为直接法和间接法。直接法直接改变轨迹的状态-控制-状态序列,同时增强状态间的模型可行性。间接法以初始状态为基础,仅对控制向量进行操作,利用动力学模型通过正向打靶获得轨迹状态。Dolgov等人\cite{dolgov2010path}提出了混合A*方法,首先用Reeds-Shepp曲线作为运动基元进行A*搜索,得到粗略的初始路径。然后用共轭梯度下降法对初始路径进行局部优化。二阶打靶方法,也叫微分动态规划\cite{6907001}是在自动驾驶领域应用广泛的一种间接法。该方法通过局部操纵控制序列来迭代地改进轨迹中的每个离散点,使用动态规划公式在当前轨迹的通道内进行优化。在每次打靶过程中考虑了运动学约束,保证轨迹是保持可行的。第四种是基于曲线的规划方法。这类运动规划方法采用插值的方式,通过考虑可行性、舒适性、车辆动力学和其他参数来拟合出局部轨迹。常用的曲线包括螺旋线、多项式曲线、样条曲线、贝塞尔曲线等。Broggi等人\cite{BROGGI2012161}使用势场将障碍物检测、沟渠定位、车道检测和全局路径规划信息的结果合并在一起,实时构建环境表示。然后在成本地图上生成基于螺旋线的运动学可行轨迹。Piazzi等人\cite{994793}将参数化的五次样条作为运动基元用于基于视觉的自动驾驶车辆控制。这种五次样条曲线具有完整性、最小性、规律性、对称性和灵活性的优点。

\subsubsection{端到端方法} 
传统的自动驾驶框架各模块是单独开发测试的。将功能模块封装后,各模块之间保留了可解释性强的接口。与之不同的是,端到端(end-to-end)的方法将感知模块与规划模块合并在一起,减少了中间部分,同时各个模块之间的接口也转为深度特征。Caltagirone等人\cite{8317618}通过整合激光雷达点云、定位信息和全局导航信息生成驾驶路径。整个方法使用卷积神经网络从真实的驾驶序列中提取感知特征。Barnes等人\cite{7989025}利用弱监督方法来分割图像中的可行驶路径。弱监督方法可以无需手动注释即可生成大量包含可行驶路径和障碍物的标记图像。这两种方法仍然可以应用在传统的规划框架中。更进一步,整个自动驾驶任务,包括车道线检测、语义抽象、路径规划和控制等,都可以整合到统一的框架中。Pomerleau\cite{Pomerleau}在1989年首次提出端到端自动驾驶驾驶框架ALVINN,采用神经网络处理摄像机图像并输出转向角度,以保持自动驾驶车辆在道路上行驶。DeepDriving\cite{7410669}在端到端的基础上将输入图像首先映射成关键性感知指标,例如车道边界检测结果等,从而增加了模型的可解释性,并且提高了驾驶性能。NVIDIA的研究人员\cite{bojarski2016end}将前置摄像头的原始图像通过深度卷积神经网络直接映射成车辆可以直接执行的转向命令。该方法能够处理挑战性的场景,例如砾石路面行驶、夜间行驶等。同时在训练时加入随机旋转和平移以增加训练样本。在隐式给出奖励函数时,数据集聚合(DAgger)通过让主要策略收集训练示例并运行参考策略来改进监督学习,从而提高主要策略的性能。Chen等人\cite{lbc}将训练分成了两个阶段,在第一个阶段通过真值数据训练出一个拥有特权的规划器,该特权规划器在第二个阶段作为监督信号训练出纯视觉的端到端规划器,最后引入DAgger算法对规划器性能进一步提高。随着自动驾驶仿真环境越来越成熟,许多基于强化学习的方法在仿真环境中得到了极大的发展。Latent DRL\cite{9346000}通过训练变分自动编码器从鸟瞰图中生成中间特征嵌入。在此基础上,Zhang等人\cite{roach}训练了一个基于强化学习的特权模型作为专家,为模仿学习智能体提供演示数据。Toromanoff等人\cite{ToromanoffWM20}提出使用由语义信息监督的隐藏状态作为强化学习策略的输入。

% ,如图\ref{fig:roach}所示
% \begin{figure}[H]
% \centering
% \includegraphics[height=5.8cm]{1_roach.pdf}
% \caption{基于强化学习的端到端框架\cite{roach}}
% \label{fig:roach}
% \end{figure}

\subsubsection{交互式运动规划方法} 
传统运动规划方法都是基于轨迹预测的结果判断自动驾驶汽车与其他交通参与者之间是否发生碰撞,并没有考虑实际交通场景涉及的自动驾驶汽车与交通参与者之间的相互影响。而这种交互影响对于保证自动驾驶汽车安全至关重要。交互式运动规划方法能够合理地推断出其他交通参与者的意图,并根据自身行为与其他交通参与者行为之间的交互影响做出最合理的决策,分为基于博弈论的方法、基于概率的方法和基于学习的方法。

基于博弈论的方法是分析交通参与者行为的有效手段。一种常见的模式是将交通参与者行为建模为预期效用最大化行为,即交通参与者会执行对自身最有收益的行为。在这种模式下,奖励函数是已知的。交通参与者固定时间范围内的所有可能行为被列举出来,然后通过奖励函数得到所有行为的得分,得分最高的行为作为最后的规划输出。Pascheka等人\cite{6873612}在车联网中引入了新的合作行为判断机制,通过计算的总效用与参考效用进行对比确定行为是否合作,然后基于自主智能体之间的通信找到最佳行为组合。Bahram等人\cite{7353203}提出了动态环境下的协同驾驶预测与规划框架,能够对场景中的所有车辆采用基于交互模型的预测并建模了所有驾驶员的重规划能力。Lenz等人\cite{7535424}提出了基于蒙特卡洛树搜索的无需车间通信的协同组合运动规划方法。Schwarting等人\cite{6957825}提出了在高速公路环境中进行协作决策的新方法,可以推断其他驾驶者的意图并整合到规划框架中得到最终合理的协同决策结果。

% 对于概率方法,这可以以类似的方式完成,其中不是针对最低成本进行优化,而是希望车辆的控制遵循最大似然法则或最大后验法则。 
运动规划问题在动态不确定性的环境中可以表述为部分观察马尔可夫决策过程。不确定性主要来源于有噪声的传感器数据以及无法直接测量的人类驾驶员意图。Hubmann等人\cite{7995949}使用POMDP确定自动驾驶汽车沿预先规划路径运动的最佳加速度策略。该策略针对其他车辆的交互式概率运动模型产生的最高概率场景进行优化,从而将周围环境的变化纳入最优策略中。Liu等人\cite{7225835}设计了一种城市道路状况环境模型,然后将状况决策问题建模为POMDP并以在线方式解决。Ulbrich等人\cite{7313257}引入了高维混合整数状态空间的规划框架,能够在换道行为规划中解决感知不确定性的问题,实现基于预测的长期运动规划。

除了端到端方法,基于学习的方法同样可以作为单独的规划模块以数据驱动的方式学习到具有交互特征的规划结果。这类方法也称为中到中(mid-to-mid)的方法。Vallon等人\cite{7995936}将决策和运动规划解耦,训练了一个具有相对位置和相对速度特征的车道变换决策支持向量机。如果触发了换道期望,则由模型预测控制器执行换道策略。ChauffeurNet\cite{chauffeurnet}通过模仿学习来训练应用到实际车辆中的自动驾驶规划策略。该方法以感知结果投影到鸟瞰图作为输入,使用CNN-LSTM网络输出规划轨迹。该方法还引入了额外的预测任务增加网络对于交互特征的提取能力。同时,ChauffeurNet通过引入额外的损失函数来增加模仿损失。这些损失会惩罚违规事件并鼓励正常行驶,从而增加了模型的鲁棒性。Huang等人\cite{dipp}提出了集成轨迹预测和规划模块的框架。该方法采用Transformer网络进行轨迹预测并利用可微非线性优化器作为运动规划器。因为整个框架都是可微的,所以能够将预测规划联合训练以获得更好的性能。

% 由于作为输入的预测方法精度不高,
\xsection{论文的主要贡献}{Major Contributions}
面向高性能、高效率、高安全性和高鲁棒性的自动驾驶应用场景,当前感知、定位和控制模块的方法大都已经基本满足需求,而规划模块仍然无法保证输出足够安全的驾驶轨迹。该问题的关键之一是作为规划输入的轨迹预测方法精度差,速度慢。大量多模态轨迹预测方法甚至无法提供其结果的概率性。这些原因导致了现有的决策规划方法难以处理与周围交通参与者之间的动态行为交互。为了解决规划模块的安全性问题,亟需提高轨迹预测方法的性能。本文主要贡献包含三部分内容,如图\ref{1_contribution}所示。本文首先提出了可以预测异质(多种类别)交通参与者单模态轨迹的时空Transformer网络,为容易保证安全性的简单场景决策规划提供高精度的输入。其次,本文提出了概率性候选轨迹网络,为安全性要求高的复杂场景决策规划提供高效快速的多模态轨迹输入。最后,本文将轨迹预测和运动规划作为共性问题处理,提出了更具前瞻性的多任务网络框架,能够同时完成轨迹预测与运动规划任务,显著地改善了自动驾驶通行能力。具体贡献描述如下:

\begin{figure}[H]
\centering
\includegraphics[width=0.98\textwidth]{1_contribution.pdf}
\caption{本文主要贡献}
\label{1_contribution}
\end{figure}

 % 将轨迹预测的思路引入低级决策和运动规划中,创造

1)提出一种基于时空Transformer网络的单模态轨迹预测网络,解决了现有方法时空特征提取能力差的问题,提高了针对异质交通参与者的单模态轨迹预测精度。

单模态轨迹预测能够为简单交通场景中规划提供安全有效的输入。传统预测方法只能保证不超过1秒的短时预测精度,无法用于长期预测。与之不同,专门用于序列数据的LSTM网络能够预测交通参与者的长期轨迹。然而,具有串行结构的LSTM网络只能使用单一深度特征向量对交通参与者的隐状态进行存储,其有限的存储能力仍然难以处理复杂的时间依赖性。另一方面,大部分预测方法通过池化机制、注意力机制或图卷积机制模拟空间相互作用力的方式,得到了空间交互特征。但是,这些方法只提取了限定空间邻域内的交互信息。这种邻域设定在自动驾驶汽车车速变化过快时会失效,导致预测精度下降。此外,大多数轨迹预测方法只能有效预测同类型交通参与者的轨迹,在处理实际多类型交通参与者同时参与的密集交通流方面存在很大的局限性。为了应对这些挑战,本文引入了具有并行数据处理能力的Transformer网络,对时空维度上的信息进行充分融合,输出了高性能的异质交通参与者单模态轨迹。
\begin{enumerate}
    \item[(a)] 本文构建了自动驾驶车辆感知范围内的全域时空图模型,并设计了时空Transformer编码器提取时空图不同维度的特征信息。在空间交互上,该编码器利用空间注意力子层的消息传递机制,将限定空间扩大至整个车辆感知范围,得到了同一时刻场景内所有交通参与者之间的空间交互信息。在时间交互上,该编码器引入时间卷积子层进行时间特征提取。通过该编码器中两种子层的交替使用,时空交互信息在不断融合的同时,生成了新的深度时空特征。
    \item[(b)] 本文设计了基于Transformer网络的编码解码框架,利用可分离卷积的深度映射能力对时空Transformer编码器产生的时空深度特征进行有效处理,输出了任意时间长度的未来轨迹。特别地,交通参与者的类别和形状等异质性特征信息会转化成额外输入,帮助网络对异质性特征进行学习,有效提高了预测精度。
    \item[(c)] 在ApolloScape轨迹数据集上,本文所提出的基于时空Transformer网络的方法在加权平均位移误差和加权最终位移误差上分别优于当时最先进的方法7.2\%和7.7\%。尤其是针对车辆的轨迹预测,本文提出方法的性能显著提高了12.21\%。% 此外,本方法已经成功在实际的自动驾驶汽车平台上部署应用。
\end{enumerate}

2)提出了一种基于概率性候选轨迹网络的多模态轨迹预测方法,在提高预测精度和加快推理速度的同时,有效解决了现有行人多模态轨迹预测方法无法提供概率性预测结果的问题。

在复杂交通场景中,单模态轨迹预测无法处理交通参与者运动模式的随机性和不确定性,带来了潜在危险性。同时输出多条合理轨迹的多模态轨迹预测方法能够为规划模块提供更具安全性的输入。大多数多模态轨迹预测方法通过从潜在分布中采样的方式,得到了具有行为不确定性的轨迹。然而,随机采样的方式无法提供所有预测结果的概率,使得规划模块无法根据不同预测轨迹的可能性大小来评估自身规划路线的代价风险,丧失了轨迹预测的实际应用价值。同时,此类方法需要大量的训练样本和复杂的神经网络结构才能够很好地处理随机性和不确定性,造成了低效性问题。与之不同,少数多模态轨迹预测方法利用启发式原则提前生成候选轨迹集,通过分类网络对每条候选轨迹进行评分,得到了概率性的预测结果。但是,这类方法的候选轨迹集质量很难得到保证,导致此类方法的性能不如基于采样的方法。本文摒弃了之前基于分类方法的启发式设计原则,通过无监督学习方式自动获取了交通参与者的候选隐意图集合。该集合经过分类网络的筛选后,高概率的多个候选隐意图会生成概率性的目标点引导信息,有效提升了预测精度。然后利用基于目标点引导信息的Transformer网络能够生成高质量的中间锚点,最后得到概率性候选轨迹。

\begin{enumerate}
    \item[(a)] 本文设计了一种有效生成目标点引导信息的新训练框架,包括目标点分布生成、聚类和分类三个步骤。首先,采用快速有效的无监督学习方法“mini-batch kmeans”自动生成训练集中交通参与者的隐意图集合。该集合能反映出交通参与者的不同运动模式。然后,引入分类网络对隐意图集合进行排序。最后,高概率的隐意图子集会映射成关于目标点的参数分布模型,从而得到概率性的目标点结果。
    \item[(b)] 本文提出了一种新的三阶段概率性多模态轨迹生成方式,在提供概率预测结果的同时,能够确保概率较高的预测结果更接近于符合交通参与者的下一步行为。第一阶段生成了概率性的目标点集合。第二阶段将目标点和历史轨迹作为输入,采用Transformer编码解码结构生成介于当前位置和目标点位置的少量中间点作为锚点。第三阶段通过连续曲线将当前位置、锚点和目标点平滑连接,得到了最终的概率性候选轨迹。
    \item[(c)] 本文所提方法比基于采样的最好方法Y-net在行人轨迹预测数据集ETH/UCY上性能高7.4\%,在斯坦福无人机数据集(SDD)上高$4.6\%$。与基于分类的最好方法PCCSNet相比,本文所提方法的性能在ETH/UCY和SDD上分别高出了$40.4\%$和$28.3\%$。同时,本文所提方法在CARLA-precog轨迹预测数据集上比当时最好的方法提高了至少20.9\%。训练和推理时间方面,本文所提方法在ETH/UCY和SDD数据集上比Y-net分别快277倍和20倍,比PCCSNet分别快28倍和3.2倍。同时,本文创新性地设计了能够证明所得隐意图集合泛化能力的迁移性实验。
\end{enumerate}

3)提出了一种基于安全轨迹树网络的运动规划方法,保证了运动规划输出具有曲率连续性和运动学可行性的轨迹,减少了现有基于学习的规划方法在完成自动驾驶任务时大量出现的交通违规行为。

轨迹预测的相关研究思路也能够拓展到具有共性的运动规划任务中。在实车中大量应用的传统运动规划方法严重依赖基于规则的启发式设计,导致其泛化能力差。大量基于规则的人工设计也给运动规划任务带来不必要的复杂性。与之相对,基于学习的运动规划方法能够极大地简化整个框架,可以分为基于强化学习和基于模仿学习两类方法。基于强化学习的方法使用奖励函数可以在无需对复杂驾驶环境进行显式建模的情况下自动获得专家级的驾驶体验。然而,设计同时满足所有运动规划约束条件的奖励函数非常具有挑战性,不当的设计会造成方法难以收敛。基于模仿学习的方法通过学习专家驾驶策略能够很好地完成驾驶任务。这类方法主要采用与轨迹预测类似的网络对专家策略进行学习。为了解决通过专家演示直接学习低级控制命令造成的过拟合问题,许多基于模仿学习的方法引入了多任务学习框架。然而,当前的多任务框架忽略了舒适性和运动学可行性等多种运动规划约束,导致输出轨迹中出现了多种违反交通规则的行为。同时,作为规划输入的栅格化深度特征可解释性低,推理速度慢。此外,这些方法无法解决自动驾驶汽车保持长期静止不动的“惯性问题”。本文采用具有曲率连续性和运动学可行性的轨迹树,能够对预测和规划这两个共性问题进行统一表征。在此基础上,本文提出了能够保证自动驾驶安全性的多任务网络框架。

\begin{enumerate}
    \item[(a)] 本文提出了一种预定义的轨迹树,能够在满足车辆运动学约束的同时显式表达出不同的车辆行为意图,为运动规划任务创建了可行的解空间。通过简单的处理改造,轨迹树也能用于预测周围环境中的交通参与者,形成了首个既用于预测又用于规划的候选轨迹集。基于这种统一的轨迹树,本文提出了能够学习预测和规划之间交互的多任务模仿学习框架。
    \item[(b)] 本文将离散化的可解释性表达方式作为空间Transformer主干网络的输入。离散化表达包括感知边界框和局部参考线。因此,空间Transformer主干网络不仅可以提取自动驾驶汽车与周围交通参与者之间的空间交互信息,而且捕捉了智能体与局部参考线之间的位置交互信息。主干网络中自动驾驶车辆的深度特征部分通过注意力机制对轨迹树上的叶轨迹进行评估,得分最高的叶轨迹经过精炼后输出规划轨迹,保证了其舒适性和运动学可行性。在训练过程中,本文采用了新的焦点损失函数,能够有效解决“惯性问题”。
    \item[(c)] 在CARLA自动驾驶仿真器的Town05测试基准上,本文所提出的方法综合驾驶性能比最先进的方法LAV提高了8.3\%,路径完成度比LAV方法提高了39.2\%。在CARLA Longest6测试基准上,本文所提出的方法的违规得分比LAV提高了10.6\%,在处理与行人碰撞和闯红灯两种违规问题的性能上比LAV分别提高了92.5\%和96.1\%。同时,本文所提出的方法在Town05和Longest6测试基准上的推理速度分别是58赫兹和28赫兹,比LAV快1.5倍,完全满足实时性的要求。
\end{enumerate}

\xsection{论文的组织结构}{Thesis Organization}
本文包含六个章节,其中3、4、5章对应上述三个主要贡献。

第1章首先概括性地介绍了自动驾驶的研究背景与意义,然后介绍了自动驾驶技术总体进展情况以及轨迹预测和运动规划的国内外研究现状。最后简要介绍了本文在提高轨迹预测精度,加快轨迹预测推理速度和保证运动规划的安全性方面做出的主要研究贡献。

第2章介绍了本文所涉及的轨迹预测和运动规划基础知识和基本方法,包括能够有效处理时空数据的编码器和解码器。

第3章提出了一种基于时空Transformer网络的单模态轨迹预测网络。本章首先分析了现有单模态轨迹预测网络主要存在的三个问题,包括空间邻域特征范围选取不合理,无法对时空交互进行精准建模和时序数据处理能力弱。针对上述问题,本章构建了时空图模型,通过时空Transformer编码器提取到全感知域内充分融合后的时空特征。同时,采用Transformer编码解码结构进一步提高了时序数据的处理能力。最后,通过定量实验、可视化展示和消融实验证明了时空Transformer网络的有效性。

第4章提出了一种基于概率性候选轨迹网络的多模态轨迹预测方法。本章首先分析了当前针对行人的大部分多模态方法无法提供概率性结果的问题,然后提出了一种三阶段概率性候选轨迹预测方法,通过设计全新的目标点引导信息生成方式,既提供了概率性的结果,又提高了性能。最后,通过数据集对比结果、计算效率分析、迁移性实验、概率分析和场景分析等大量实验对概率性候选轨迹网络的性能进行了全面展示。

第5章提出了一种基于安全轨迹树网络的运动规划方法。本章首先分析了各种运动规划方法的局限性,重点介绍了现有基于学习的方法存在的大量违章现象。然后观察到传统规划方法在数据输入和运动学可行轨迹输出方面的优势,进而将传统规划方法的低维数据输入和螺旋曲线引入基于学习的方法中,创造了轨迹树模板。最后,在自动驾驶闭环测试基准上验证了安全轨迹树网络的性能。

第6章总结了全文的主要内容和贡献,并对未来值得进一步探索的工作和具有潜力的方向进行了展望。


