% !TeX root = ../main.tex

\xchapter{结论与展望}{Conclusion and Future Work}

\xsection{结论}{Conclusion}
本文围绕“多光谱融合智能光电处理算法与系统设计”这一主题,针对无人机在低空复杂环境中对实时、鲁棒感知与跟踪的迫切需求,展开了一系列从理论方法、关键算法到工程系统的深入研究。现有目标检测跟踪算法在无人机视角下仍存在技术瓶颈,对于像素占比极小的目标,其检测精度仍有较大提升空间,对于长时被遮挡目标,现有目标跟踪算法缺乏可靠的丢失判定与重检测机制。本文通过将先进的人工智能算法与严格的嵌入式工程约束相结合,致力于解决机载平台在有限算力、内存和功耗下实现高精度环境感知的挑战,构建了一套从核心处理算法到完整软硬件系统的完整解决方案。本文主要研究成果包括以下四个方面。
\begin{enumerate}
    \item[(1)] 提出了面向可见光航拍图像的高性能小目标检测网络BAP-DETR,在无人机航拍图像的目标检测任务中,算法需应对极端尺度变化、密集小目标不均匀分布及复杂背景干扰三重挑战。通用目标检测器在此类场景下存在显著性能差距,而直接提高输入图像分辨率又会引发计算量激增,难以满足机载平台的实时性要求。现有小目标检测方法的网络架构往往在特征传递过程中丢失对小目标至关重要的细粒度信息,导致精度与速度难以兼顾。针对这些问题,本文设计了双重注意力处理模块,通过通道分离策略实现卷积与自注意力全局建模能力的并行优化与交互,使模型能从复杂场景中提取更具判别力的细微特征。其次,设计了配备频域感知融合模块的双融合编码器,该设计能有效保留并融合包含丰富细节的低层特征与承载语义信息的高层特征,显著增强了对小目标的特征保留能力与多尺度感知性能。最后,在损失函数中,结合倒数归一化Wasserstein距离与CIoU损失,在不增加推理开销的前提下,进一步提升了对小目标定位的精确度与鲁棒性。在VisDrone、UAVDT和AI-TOD三个公开航拍数据集上的实验表明,BAP-DETR在保持推理效率的同时,平均检测精度(AP)较基线模型提升6.9\%,并实现了17.5\%的计算负载降低。这项工作为无人机可见光视角下的实时高精度小目标检测提供了一个有效的解决方案,有效平衡了机载场景下对精度与速度的要求。

    \item[(2)] 设计了面向无人机红外图像的轻量化小目标检测网络MFF-DCNet,无人机红外成像在夜间、烟尘等恶劣环境下具有不可替代的优势,但其图像固有的低分辨率、低对比度、高噪声及纹理缺失特性,使得目标检测,尤其是对远距离像素占比极小的微弱目标的检测,成为极具挑战性的难题。现有通用检测模型直接迁移至红外图像时性能显著下降,而计算密集的先进网络又难以在机载边缘设备的严格算力约束下实现实时推理。针对这些问题,本文设计了深度可分离跨阶段Transformer模块用于增强主干网络的特征提取能力,该结构有效建模了远距离弱小目标与复杂背景之间的长距离上下文依赖关系,增强了特征的判别性。其次,提出了一个新颖的多特征聚焦颈部结构,通过自适应的跨尺度特征加权与融合策略,提升了网络对小目标特征的提取能力。在HIT-UAV和DroneVehicle红外航拍数据集上的实验表明,MFF-DCNet不仅检测精度(AP)显著超越专用无人机图像检测器及YOLO、DETR系列等基线模型,更在处理效率上实现了提升。同时,该网络在NVIDIA Jetson Orin NX嵌入式边缘计算平台上达到了39.6 FPS的实时处理能力,验证了其满足实际机载任务对低功耗、高实时性的要求,为无人机全天候智能感知提供了可靠的红外视觉解决方案。

    \item[(3)] 提出了一种面向边缘计算设备的抗遮挡长时目标跟踪框架SKF-Tracker,在无人机执行持续监视、目标跟踪等任务时,目标被环境中的建筑、植被等障碍物完全遮挡是导致跟踪失败的主要原因。现有主流跟踪算法及公开数据集多聚焦于短时、部分遮挡的场景,且缺乏无人机视角下的图像,导致算法在实际复杂城市场景中鲁棒性不足。针对这一问题,本文设计了一种具备目标丢失判断与重捕获能力的抗遮挡长时跟踪框架,使用结构相似性指数作为跟踪置信度判断依据,结合轨迹预测与动态搜索的重捕获机制,使系统能够在识别到目标被遮挡的情况后,及时暂停目标模板更新,在目标重现时快速锁定,同时,自适应的模板更新策略确保了模型外观记忆的可靠性。为了针对性地验证算法抗遮挡能力,本文构建了一个多模态(可见光与红外)无人机视角目标跟踪数据集MMUOT-1050,包含353段可见光视频序列与697段红外视频序列,每段视频均包含目标被完全遮挡的场景。SKF-Tracker在该数据集上实现了89.37\%的可见光视频成功率和91.93\%的红外视频成功率,与基线方法相比提升了14\%和11.73\%。在实时性方面,SKF-Tracker在保持最高精度的同时,仍能保持31.25的帧率,其速度远超基于深度神经网络的SiamRPN,并且不占用NPU资源。该工作为解决面向边缘计算设备的抗遮挡目标跟踪提供了一套兼具理论创新与工程落地的可行路径。
    
    \item[(4)] 设计并实现了一套能够满足低功耗、高实时性要求的机载智能光电原型系统,将算法创新融入到完整的工程实践中。该系统采用高性能边缘计算模组,集成了模块化的多光谱数据采集、处理与通信软件框架,并通过大疆M350 RTK无人机平台进行了实地飞行验证。在硬件层面,系统以高性能异构边缘计算模组(RK3588)为核心,集成了可见光与红外传感器,实现了多光谱数据的采集与处理。利用芯片厂商提供的专用工具链(RKNN),对网络模型进行了针对性的优化,解决了网络模型在边缘侧部署的核心瓶颈。在软件层面,本文构建了一套模块化、高内聚、低耦合的嵌入式应用软件框架。数据接入层以工厂模式管理多源异构传感器,数据处理层以策略模式封装并动态调度检测、跟踪等智能算法,通信总线层以事件驱动与消息队列实现模块间异步通信,通过无锁环形缓冲区和内存池保障了数据流的高效与确定性。同时,系统提供了支持多协议的上位机交互接口并支持算法参数高度可配置,大幅提升系统的可扩展性与工程可维护性。
\end{enumerate}

本文围绕低空无人机在复杂环境下的智能感知需求,从理论方法、核心算法、系统实现到实验验证,实现了从算法设计到工程实现与性能验证的完整闭环。所提出的BAP-DETR、MFF-DCNet及SKF-Tracker算法,针对无人机视角下的小目标检测与严重遮挡跟踪等难题提供了有效解决方案,基于软硬件协同设计的智能光电原型系统,为算法的工程化落地与性能验证提供了坚实基础。本工作不仅在提升无人机自主感知的精度、鲁棒性与实时性方面取得了进展,同时为机载智能光电系统原型设计提供了经验与参考。

\xsection{展望}{Future Work}
本文围绕无人机智能光电处理算法与系统设计展开了一系列研究工作,取得了显著成果。但仍有许多值得深入探索的方向,未来工作将围绕以下几个方面展开:

\paragraph{基于深度聚类的多模态轨迹预测方法研究}
本文所提出的多模态轨迹预测方法研究使用了以“k-means”为基础的无监督学习方法尽可能精准地收集交通参与者的隐式运动模式。但是“k-means”聚类方法需要预先设置聚类的类别数量,而以“DBSCAN”为核心的其他传统聚类方法虽然不用设置类比数量的超参数,但是其针对交通参与者生成的隐特征聚类质量明显劣于“k-means”聚类方法。近年来,大量的深度聚类方法涌现出来。其核心思想是通过表示学习和聚类之间相互提升,良好的表示改进聚类的质量,而良好的聚类会为表示学习提供良好的监督信号。因为深度聚类方法不仅对数据分布的假设更少,而且通常能够产生更好的无监督学习结果,所以可以将需要超参数设置的传统“k-means”聚类方法转变为深度聚类方法,从而自动从轨迹预测数据集大量的数据中提取到丰富的运动行为模式。采用深度聚类方法的另一个优势是可以和多模态轨迹预测网络联合训练,减少生成过程的中间步骤,并通过预测误差的后向传播的方式进一步提升聚类质量。

\paragraph{基于高精度地图信息的轨迹预测方法研究}
本文所提出的方法考虑了目标指导信息的利用、交通参与者自身的时间交互和交通参与者之间的空间交互,忽略了交通参与者与场景之间的交互特征。考虑交通参与者与场景之间的交互特征可以为轨迹预测方法设计引入更多的信息。在搭载激光雷达和相机的实际自动驾驶系统中,利用感知模块和定位建图模块能够构建出包含车道线、停止线、斑马线、红绿灯及其他交通标识的高精度语义地图。在规划模块中,包含丰富语义信息的局部参考线就是根据高精度语义地图生成的。自动驾驶汽车能够根据局部参考线提供的地图信息及周围动静态目标信息精准还原出其当下所处的真实环境。这种规划模块的处理思路完全可以引入针对自动驾驶汽车周围动态交通参与者的轨迹预测中。在进行轨迹预测时,可以根据高精度地图提供丰富语义信息,构建出感知范围内所有交通参与者其自身所处的不确定性环境。后续工作将考虑以栅格地图或者矢量化地图的形式引入高精度地图语义信息,并采用合适的网络模型进行不确定性估计和特征提取。

\paragraph{强化学习和模仿学习相结合的自主决策规划算法研究与应用}
本文提出的轨迹预测与规划一体化框架能够完成CARLA仿真器闭环测试环境下的多种挑战性任务路线。但是,通过空间Transformer编码器产生的决策部分仍然非常简单,只能够完成跟车以及固定地点的汇入车流和换道等简单决策,并不能够做出自由换道等高级决策。深度强化学习能够有效应对自动驾驶多种复杂场景下的决策。然而,当前深度强化学习的方法同样存在样本复杂性高和稳定性弱的问题。具有专家知识的模仿学习方法可以弥补上述缺陷。因此后续工作考虑将来自强化学习探索方面的优势与模仿学习在学习专家策略方面的能力相结合,寻求采用易于实现的通用强化模仿学习方法完成自动驾驶自主决策规划任务。模仿学习训练得到的专家策略能够为深度强化学习提供性能优越的初始化智能体,而深度强化学习通过深入探索,可以突破专家策略的局限,如专家选择的低效保守的跟车策略等。这种强化学习和模仿学习相结合的方式对于进一步提升甚至是跨越式提升基于学习的规划方法在闭环测试中的性能具有重要研究价值。

% 西安交通大学“发现号”
\paragraph{多模态轨迹预测及运动规划方法在实际自动驾驶平台的部署}
本文所提出的多模态轨迹预测方法仍然只是在数据集上进行了验证。未来工作考虑将其部署在实车平台上,真正从实际应用的角度去发现和解决多模态轨迹预测问题。同时,现有的大部分实际自动驾驶平台中使用的规划方法仍然无法很好地处理多模态的轨迹预测输入。如百度Apollo开源版本的规划方法只选择了概率最大的多模态轨迹预测结果作为规划输入,相当于转化为了单模态的轨迹预测输入。因此,如何更好地处理多模态轨迹预测输入对于实际自动驾驶汽车具有重要意义。另一方面,本文所提出的规划预测一体化方法仍然停留在仿真测试阶段,其选择的专家数据集也是从CARLA仿真器中采集的。打破模仿学习和强化学习方法从仿真环境到现实环境瓶颈的“sim-to-real”研究方向是当前基于学习的规划方法落地的重要保证之一。未来工作拟结合之前在实车平台积累的研发优势,为基于学习的规划方法突破实用化限制开展相关研究工作。

